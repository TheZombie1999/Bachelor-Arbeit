\chapter*{Symbolverzeichnis}
\addcontentsline{toc}{chapter}{Symbolverzeichnis}
Einträge werden an beliebiger Stelle erzeugt mit \begin{verbatim}
    \newglossaryentry{c}{
    name=\ensuremath{c},
    description={Lichtgeschwindigkeit},
    unit={\SI{299792458}{\metre\per\second}},
    type=konstante}
\end{verbatim}
Als type lässt sich entweder \textit{symbol} oder \textit{konstante} wählen. 
Im Header können die Werte von \verb+\dimen0+ und \verb+\dimen2+ angepasst werden, um die Spaltenbreite der ersten und letzten Spalte zu ändern.


%%%%%%%%%%%%%%%%%%%%%%%%%%%%%%%%%%%%%%%%%%%%%%%%%%%% 
% Glossary Einträge
   
 \newglossaryentry{c1}{
    name=\ensuremath{c},
    description={Lichtgeschwindigkeit},
    unit={\SI{299792458}{\metre\per\second}},
    type=konstante}
    
    \newglossaryentry{c2}{
    name=\ensuremath{c},
    description={Lichtgeschwindigkeit Lichtgeschwindigkeit Lichtgeschwindigkeit Lichtgeschwindigkeit Lichtgeschwindigkeit Lichtgeschwindigkeit},
    unit={\SI{299792458}{\metre\per\second}},
    type=konstante}

	\newglossaryentry{leistung1}{
    name=\ensuremath{P},
    description={Leistung},
    unit={\si{kW}},
    type=symbol}

    \newglossaryentry{leistung2}{
    name=\ensuremath{P},
    description={Leistung Leistung},
    unit={\si{kW}},
    type=symbol}

    \newglossaryentry{leistung3}{
    name=\ensuremath{P},
    description={Leistung Leistung Leistung Leistung Leistung Leistung Leistung Leistung Leistung Leistung Leistung},
    unit={\si{kW}},
    type=symbol}
    
%%%%%%%%%%%%%%%%%%%%%%%%%%%%%%%%%%%%%%%%%%%%%%%%%%%%%%
% zum Ausgeben der Verzeichnisse 
\symbolverzeichnis

