% TODO
% Quellen
% Englische Worte Kursiv
% Rechtschreibung und Gramatik

\section{ Fehlerbehebung in den impliziten Lösern } \label{sec:fehler_in_implizit_solver}

Obwohl die \textit{scatter}-Funktion nun \textit{Dual}-Zahlen auf der Grafikkarte verarbeiten kann, 
funktionieren die implizierten Löser noch nicht. 
Julia hat sich im Laufe der Zeit weiter entwickelt und einig API Änderungen vorgenommen.
Dies hat dazu geführt, dass die \textit{lu-instance} Funktion
der \textit{ArrayInterface.jl} Bibliothek \cite{arrayinterface} nicht mehr aktuell
war.
Das Problem war, dass sich die Implementierung der LU-Zerlegung geändert hat.
In den älteren Versionen hat \textit{CUDA} \cite{besard2018juliagpu} diese Implementierung selbst bereitgestellt.
Mit der Version 1.8 wurde diese Funktion so in die Standardbibliothek mit aufgenommen, 
dass die Faktorisierung auch auf der Grafikkarte durchgeführt werden kann.
Dank der Hilfe von Christopher Rackauckas \cite{chris} ist es 
dann nach vielen Versuchen gelungen, die \textit{lu\_instance} Funktion 
von \textit{ArrayInterface.jl} richtig umzusetzen \cite{firstIssue,finalIssue,testForIssue,rosenbrock}
und es damit möglich zu machen, implizite Löser zu verwenden.