\section{ Fehler behebung in den Impliziten Lösern }

Da nun die scatter Funktion mit \textit{dual} zahlen auf der GPU durchgeführt werden kann, 
können prinzipiel implizite Differentialgleichungs löser für 
NeuralODEs auf der Graphikkarte verwendet werden.

Julia hat sich im laufe der zeit weiter entwickelt und einig API Änderungen vorgenommen.

Dies hat auf der Seite der Löser dazu geführt, dass diese zum Teil Funktionalität von Julia 1.8 wollten 
und zum Teil Funktionalität von den früheren Versionen.
Das Problem war das sich die Implementierung der lu-zerlegung geändert hat. % lu-zerlegung
In den älteren Versionen hat CUDA diese Implementierung selbst bereitgestellt.
Mit der Version 1.8 wurde dies allerdings in die Standard Bibliothek mit aufgenommen, das die Faktorisierung 
auch auf der Grafik karte durchgeführt werden kann.
Das OrdenaryDifferentialEquations packet von Julia verwendet 
ein Array-Interface, welches operationen auf Arrays, 
von der Umsetzung abstrahiert, um diese auf einer Grafikkarte durchführen zu können.
Dank der Hilfe von Christopher Rackauckas ist es, dann nach vielen versuchen gelungen, die lu\_instance Funktion, von \textit{ArrayInterface.jl} richtig umzusetzen 
und es damit möglich zu machen zumindest die im nächsten Kapitel vorgestellten impliziten Löser zu verwenden.
