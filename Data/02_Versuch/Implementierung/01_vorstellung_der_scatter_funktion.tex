\subsection{Vorstellung der Scatter funktion}

Um die Probleme der \textit{scatter} funktion lösen zukönnen, 
wird dessen funktionsweiße vorgestellt.
Die Aufgabe der \textit{scatter} function ist es werte aus einem N-Dimensinalen Array in ein anders 
ähnlicher Größe Array zuschreiben und dabei auf alle in eine zelle geschriebenen werte eine Operation anzuwenden.
Es können zum beispiel mehrere eingaben in einen Knoten eines Graphen, 
welcher in Matrix form aufgestellt wurde,
zusammen gerechnet werden.
Die besonderheit der scatter function ist dabei, 
das dies nicht nur für eine Zelle geschieht, 
sonder für alle.
Die \textit{scatter} function ist wie folgt definiert:

\begin{lstlisting}{language=Julia}
	"""
	op ist die operation die auf alle werte die in die selbe zelle geschrieben werden angewendet wird.
		mögliche werte sind +, -, *, /, min, max, mean
	src ist eine beliebige matrix aus der wert gelesen und wieder an anderer stelle zurück geschrieben werden.
	idx ist ebenfalls eine beliebige matrix aus der entnommen wird an welche position, der wert einer zelle geschrieben werden soll.
	init beschreibt wie das ziel array initialisiert wird.
			dies kann zum beispiel genutzt werden, um den inhalt des arrays in das geschrieben wird initial auf null zu setzten.
	dstsize gibt die größe des zurück gegebenen arrays an.
	"""

	function scatter(op, src, idx; [init, dstsize])
		dst = similar(src)
		
		fill!(dst, init)
		
		scatter!(op, dst, src, idx)
	end
\end{lstlisting}

Die implementierung der \textit{scatter} funktion ist eine vereinfachte darstellung, 
welche details zur berechnung der größe des dst array weg lässt.
Die gesamte implementierung kann in \scatter{NNlib.jl} \cite{} geschlagen werden.

Wichtig an der Implementierung ist, dass die \textit{similar} funktion verwendet wird um, das
\textit{dst} Array zu erstellen.

Dieser Aufruf stellt sicher das \textit{dst} und \textit{src} den selben datentyp haben.

Dies hat den Großen vorteil, 
dass die beiden Arguemente der \textit{op} Operation von Gleichen Datentyp sind. 
Dies verhindert, 
dass viele spezialfälle der addition von Dualzahlen, nicht implementiert werden müssen.

Es fällt ausserdem auf das die Implementierung der berechung innheralb der \textit{scatter!}
durchgeführt wird.


Das Ausrufe zeichen im Funktions namen, bedeutet das die argumente der Funktion überschrieben werden können.


Die \textit{NNlibCuda.jl} Bibliothek überlädt diese Funktion für \textit{CUDA}-Arrays,
implementiert die berechnung auf der Graphikkarte.

Innerhalb dieser implementierung tritt der Fehler auf der die Verwendung von \textit{Dual}-Zahlen,
verhindert.
