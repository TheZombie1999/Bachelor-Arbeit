\subsection{Vorstellung der \textit{scatter}-Funktion} \label{sec:vorstellung}

Um die Probleme der \textit{scatter}-Funktion lösen zu können, 
wird in diesem Kapitel erstmal deren Funktionsweise betrachtet.
Aufgabe dieser Funktion ist es, den Wert aus einer Zelle 
eines mehrdimensionalen Arrays, in ein anderes Array ähnlicher Größe zu schreiben.
Dabei wird ein Operator auf alle Werte, die in dieselbe Zelle
geschrieben werden, angewendet.
Dies findet zum Beispiel Anwendung, wenn alle Eingaben in einen Knoten eines Graphen zusammen gezählt werden sollen.
Der Graph besteht dabei aus einem Vektor aus Koten 
und einer Matrix aus Kanten.
Die Besonderheit der \textit{scatter}-Funktion ist dabei, 
dass dies nicht nur für eine Zelle geschieht, sondern für alle.
Wird diese auf der Grafikkarte ausgeführt, passiert dies parallel.
Die Definition der \textit{scatter}-Funktion lautet wie folgt:

\begin{lstlisting}{language=Julia}
	"""
	op: ist die Operation die auf alle Werte die in die selbe Zelle geschrieben werden, angewendet wird. Moegliche Werte sind +, -, *, /, min, max, mean.
	src: ist eine Matrix aus der Werte gelesen und wieder an anderer stelle zurueck geschrieben werden.
	idx: ist ebenfalls eine Matrix aus der entnommen wird, wo der Wert einer Zelle hin geschrieben werde soll.
	init: beschreibt wie das Ziel-Array initialisiert wird. Dies kann zum Beispiel genutzt werden, um den Inhalt des Arrays in das geschrieben wird, initial auf null zu setzten.
	dstsize: gibt die Groesse des zurueck gegebenen Arrays an.
	"""

	function scatter(op, src, idx; [init, dstsize])
		dst = similar(src)
		
		fill!(dst, init)
		
		scatter!(op, dst, src, idx)
	end
\end{lstlisting}

Die Implementierung der \textit{scatter}-Funktion ist eine vereinfachte Darstellung, 
welche Details zur Berechnung der Größe des \textit{dst}-Arrays weglässt.
Die gesamte Implementierung kann in der \scatter{NNlib.jl} Bibliothek \cite{nnlib} nachgeschlagen werden.
Wichtig an der Implementierung ist, dass die \textit{similar}-Funktion verwendet wird, um das
\textit{dst}-Array zu erstellen.
Dieser Aufruf stellt sicher, dass das \textit{dst}- und \textit{src}-Array denselben Datentyp haben.
Dies hat den großen Vorteil, 
dass die beiden Argumente der \textit{op} Operation von gleichem Datentyp sind.
Dies verhindert, 
dass viele Spezialfälle der Addition von Dualzahlen nicht implementiert werden müssen.
Es fällt außerdem auf, dass die Implementierung der Berechnung innerhalb der \textit{scatter!}-Funktion
durchgeführt wird.
Das Ausrufezeichen im Funktionsnamen bedeutet, dass die Argumente der Funktion überschrieben werden können.
Die \textit{NNlibCuda.jl} \cite{nnlibcuda} Bibliothek überlädt diese Funktion für \textit{CUDA}-Arrays und
implementiert die Berechnung auf der Grafikkarte.
Innerhalb dieser Implementierung tritt der Fehler auf, der die Verwendung von \textit{Dual}-Zahlen
verhindert.