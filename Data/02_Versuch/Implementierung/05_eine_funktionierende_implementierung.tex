\subsection{Eine Funktionierende Implementierung}

In diesem Kaptipel geht es darum, wie die \textit{scatter} function umgesetzt werden kann.

Dazu werde die argumente so umgeformt, 
dass die bereits vorhandenen implementierungen der \textit{scatter!} function, 
diese richtig verarbeiten können.

Es wird dafür zurerst \textit{Global Memory} \ref{memory} reserviert und dann werden Kernels verwenden, 
um die entsprechenede werte in diesen zu kopieren.

Ein kernel ist eine Function die Auf der Graphikkarte ausgeführt wird.

Wichtig ist dabei das der CUDA-speicher die richtige Dimension haben. 

Wichtig ist dabei, dass die Kopierten Arrays die richtige dimension und inhalt haben. 

Die ursprünglichen Arrays \textit{src} und \textit{dst} haben die Dimension N.

Das Array \textit{idx} entweder dimension N und enthält tupel von gleicher Größe,
oder dimension 1 und und enthält \textit{Interger} Zahlen.

Falls die Dimension von \textit{idx} Eins ist, und die Dimension von \textit{src}
und \textit{dst} größer dann wird \textit{idx} zeilen weiße angewendet.

Es wird nun eine Kopie dieser Drei Arrays erstellt mit Dimension N+1.

In der Extra Dimension werden die einzelnen Komponenten der Dual Zahl gespeichert.

Ein Beispiel für $N = 1$, sieht für \textit{src} und \textit{dst} sieht wie folgt aus:

\begin{equation}
\text{src oder dst} = 
\begin{matrix}
Dual & Dual & Dual\\
\end{matrix}
\rightarrow
\text{src\_m oder dst\_m} = 
\begin{matrix}
Value   & Value   & Value   \\
Partial & Partial & Partial \\
Partial & Partial & Partial \\
Partial & Partial & Partial \\
\end{matrix}
\end{equation}

Das \textit{idx} Array ist in dieser hinsicht besonders.
Da die einzelnen \textit{Integer}-Zahlen zu Tupel um gewandelt werden müssen. 

\begin{equation}
\text{idx}= 
\begin{matrix}
 2 & 1 & 3
\end{matrix}
\rightarrow
\text{idx\_m} = 
\begin{matrix}
 (2, 1) & (1, 1) & (3, 1) \\ 
 (2, 2) & (1, 2) & (3, 2) \\
 (2, 3) & (1, 3) & (3, 3) \\
 (2, 4) & (1, 4) & (3, 4) \\
\end{matrix}
\end{equation}

Die Anzahl der Zeilen in beiden Beispielen, ist abhänig von der anzahl der Freien variablen  
der Multidimensionalen Dual Zahlen.

Wird nun \textit{dst\_m}, \textit{src\_m} und \textit{idx\_m} an die \textit{scatter!} function über
geben, wird das erwünschte ergebnis berechnet. 

Es muss nur noch in das korekte format zurück kopiert werden. 

Für diesen Letzten schritt müssen mehrere \textit{views} auf
\textit{dst\_m} erstellt werden.

Das verwenden von \textit{views} Löst das Problem von scalarer indezierung innerhalb der Graphikkarten Kernel. 

Eine \textit{view} erlaubt es die art und weiße wie auf einen speicher bereich zugegriffen wird zu verändern.

Diese werden auf der CPU erstellt und dann dem Kernel als argument übergeben.

Um zum Beispiel ein Array von Tupeln, mit den Partials als inhalt zuerhalten, wird folgender Code ausgeführt:

\begin{verbatim}
	function f(x...)
		return x
	end
	
	tuples = f.([view(dst_m, repeat([:], 
	        ndims(dst_m) - 1)..., i) for i in 2:(N+1)]...)
\end{verbatim}


Für mehre details zur genauen umsetzung, kann der Code im anhang eingesehen werden.

