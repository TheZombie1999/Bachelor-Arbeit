\subsection{Memory Allocation}

Wie bereits im Kaptiel \ref{atomic} beschrieben, muss die addition von Dual zahlen komponenten weiße durchgeführt werden.
Da es aber nicht möglich ist auf die einezelen komponentn direkt zuzugreifen müssen diese zwischen gespeichert werden um direkten zugriff zu erlauben.
Der erste ansatzt, dazu ist speicher direkt im kernel zu reservieren.
Dies scheitert daran, dass andere Threads auf diesen speicher nicht zugreifen können.
Um speicher zu erstellen der von allen threads verwendet werden kann, kann shared memory verwendent werden.
Das problem dabei ist das shared memory nur von threads im gleichen thread block sind zugegriffen werden kann.
Es kann global speicher reserviert werden, indem dieser Ausserhalb des Kernels allociert und dann beim start diesem als arguemnt übergeben wird.
Dieser Ansatz führt zu erfolg, indem ein eigenenr speicher bereich für jede komponente einer dual zahl reserviert wird, den entsperechenden wert in diesen hinein copiert
und dann die addition analog zu \ref{} innerhalb der \textit{scatter!} function implementiert.
Dieser Ansatz hat zwei probleme.
Das erste Problem ist, dass große teil der bereits vorhandenen implementierungen der \textit{scatter!} function mit minimalen änderungen reimplementiert werden müssen, was 
gegen das dry-Prizip verstößt.
Zweitens ist es aktuell zwar möglich die berechnung korrekt durchzuführen, dass ergebniss kann aber noch nicht in das ziel array zurück geschrieben werden.
Das nächste Kaptiel bespricht wie genau diese zwei Probleme gelößt werden können.

