\subsection{Speicherreservierung auf der Grafikkarte} \label{sec:memory}

Wie bereits in Kapitel \ref{sec:atomic} beschrieben, 
muss die Addition von \textit{Dual}-Zahlen komponentenweise durchgeführt werden.
Da es aber nicht möglich ist, auf die einzelnen Komponenten direkt zuzugreifen,
müssen diese zwischen gespeichert werden, um direkten Zugriff zu erlauben.
\textit{CUDA} \cite{besard2018juliagpu} unterstützt die folgenden Arten von Speicher auf der Grafikkarte:

\begin{itemize}
	\item Constant Memory
	\item Local Memory
	\item Shared Memory
	\item Global Memory
\end{itemize}

\textit{Constant Memory} ist für die Umsetzung der \textit{scatter!}-Funktion nicht geeignet,
da dieser nicht verändert werden kann.
\textit{Local Memory} ist Speicher, 
auf den über die Register eines Threads in der Grafikkarte zugegriffen wird.
Die Verwendung von \textit{Local Memory} scheitert daran, 
dass nur ein Thread auf diese Register zugreifen kann.
\textit{Shared Memory} ist ein Speicher, 
der von allen Threads in einem Threadblock erreichbar ist.
Da nicht sichergestellt werden kann, dass alle Threads, 
die in die selbe Speicherstelle schreiben, auch im selben Threadblock sind,
kann dies nicht verwendet werden.
Der Speichertype für unser Problem ist \textit{Global Memory}, 
da dieser von allen Threads erreicht werden kann,
mit dem Nachteil, dass die erhöhte Geschwindigkeit der andere Speichersorten
nicht genutzt werden kann.
\textit{Global Memory} muss außerhalb des Kernels allokiert werden, 
was zu einer Fragmentierung der Umsetzung der Addition von \textit{Dual}-Zahlen führt.
Die Fragmentierung führt dazu, 
dass große Teil der bereits vorhandenen Implementierungen der \textit{scatter!}-Funktion mit minimalen Änderungen noch einmal
implementiert werden müssen.
Die Verwendung von \textit{Global Memory} ermöglicht es, die Berechnung korrekt durchzuführen,
das Ergebnis kann aber noch nicht in das Ziel Array zurückgeschrieben werden.
In dem nächsten Kapitel geht es darum, wie die letzten zwei Probleme gelöst werden können.
