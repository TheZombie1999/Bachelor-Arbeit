\subsection{\textit{Dual}-Zahlen} \label{sec:dual_zahlen}

Um zu verstehen, was auf der Grafikkarte zu Problemen führt, 
wird zunächst die Definition von \textit{Dual}-Zahlen betrachtet.
Diese werden von der \textit{ForwardDiff.jl} Bibliothek \cite{juliaForwardDiffPackage}
implementiert. 
Die Definition dieser Zahlen ist von Imaginären-Zahlen abgeleitet.
Genauer gesagt handelt es sich um mehrdimensionale \textit{Dual}-Zahlen, 
da sich hinter diesem Begriff zwei sehr ähnliche Konzepte verbergen, 
werden folgende zwei Begriffe eingeführt.
Mit einer \textit{Dual}-Zahl wird die konkrete Implementierung in der \textit{ForwardDiff.jl} Bibliothek \cite{juliaForwardDiffPackage} bezeichnet.
Eine multidimensionale \textit{Dual}-Zahl bezeichnet die mathematische Definition, 
welche von imaginären Zahlen abgeleitet ist.

\begin{equation}
 d = x + \sum_{i = 1}^{k} y_i \epsilon_i 
\end{equation}

Dabei ist d eine multidimensionale \textit{Dual}-Zahl mit der reellen Komponente $x$ 
und den partiellen Komponenten $y_1, ..., y_k$.
Dabei sind die Variablen $\epsilon_1, ..., \epsilon_k$ ähnlich zu der imaginären Zahl $i$ mit dem Unterschied, dass diese folgende Eigenschaft haben:
\begin{equation}
 \epsilon_i \cdot \epsilon_j = 0
\end{equation}

Für die Definition von multidimensionalen \textit{Dual}-Zahlen \cite{juliaForwardDiffPackage}
ist ebenfalls ihr Verhalten auf skalare Operationen wichtig:

\begin{equation}
 f( x + \sum_{i = 1}^{k} y_i \epsilon_i  ) = f(x) +  f'(x) \sum_{i = 1}^{k} y_i \epsilon_i
\end{equation}

Das Verhalten auf skalare Operationen ist, entscheidend 
für die Berechnung von automatischen Differenzierungs-Algorithmen.
Um die \textit{scatter}-Funktion umzusetzen, 
wird die Definition der Addition von multidimensionalen \textit{Dual}-Zahlen \cite{RecentAdvances} benötigt.

\begin{equation}
(x_1 + \sum_{i = 1}^{k} y_{1,k} \epsilon_i) + ( x_2 + \sum_{i = 2}^{k} y_{2,k} \epsilon_i) = (x_1 + x_2) + \sum_{i = 1}^{k} (y_{1, i} + y_{2, i}) \epsilon_i
\end{equation}
Alle weiteren Operationen lassen sich ebenfalls von den komplexen Zahlen ableiten, 
sind aber nicht relevant, um die Differenzialgleichungslöser zu unterstützen.
