\subsection{Dual Zahlen}

Um zu verstehen, was auf der Grafikkarte zu Problemen führt, 
wird zunächst die Definition von \textit{Dual}-Zahlen betrachtet.
Da sich hinter diesem Begriff zwei sehr ähnliche Konzepte verbergen, 
führen wir folgende zwei Begriffe ein.
Mit einer \textit{Dual}-Zahl wird die konkrete Implementierung in der \textit{ForwardDiff.jl} Bibliothek bezeichnet.
Eine Multidimensionale \textit{Dual}-Zahl bezeichnet die mathematische Definition, 
welche von imaginären Zahlen abgeleitet ist.

\begin{gather*}
 d = x + \sum_{i = 1}^{k} y_i \epsilon_i \\
	\text{ d ist eine Multidimensionale Dual Zahl } \\
	\text{ x ist die relle componente } \\
	y_i \text{ sind die imaginären komponenten mit } i \in \{1, ..., k\}
\end{gather*}

Bei Multidimensinalen \textit{Dual}-Zahlen git:
\begin{equation}
 \epsilon_i \cdot \epsilon_j = 0
\end{equation}

Für die Definition von Multidimensinalen \textit{Dual}-Zahlen \cite{juliaForwardDiffPackage}
ist ebenfalls ihr verhalten auf scalare Operationen wichtig:

\begin{equation}
 f( x + \sum_{i = 1}^{k} y_i \epsilon_i  ) = f(x) +  f'(x) \sum_{i = 1}^{k} y_i \epsilon_i
\end{equation}

Das verhalten auf skalare Operationen ist entscheidet 
für die Berechnung von automatischen Differenzierungs Algorithmen.

Um die \textit{scatter}-Funktion umzusetzen, 
wird die Definition der Addition von Multidimensionalen \textit{Dual}-Zahlen \cite{RecentAdvances} benötigt.

\begin{equation}
(x_1 + \sum_{i = 1}^{k} y_{1,k} \epsilon_i) + ( x_2 + \sum_{i = 2}^{k} y_{2,k} \epsilon_i) = (x_1 + x_2) + \sum_{i = 1}^{k} (y_{1, i} + y_{2, i}) \epsilon_i
\end{equation}
Alle weiteren Operationen lassen sich ebenfalls von den von den komplexen Zahlen ableiten, 
sind aber nicht relevant, um die Differenzialgleichungslöser zu unterstützen.
