\subsection{Dual Zahlen}

Um zu verstehen, was auf der Graphikkarte zu Problemen führt, wird zunächst die Definition von \textit{Dual}-Zahlen betrachtet.

Da sich hinter diesem Begriff zwei sehr ähnliche Konzepte verbergen führen wir folgende zwei begriffe ein.

Mit einer \textit{Dual}-Zahl wird, die konkrete implementierung in der \textit{ForwardDiff.jl} Bibliothek bezeichnet.

Eine Multidimensionale Dual zahl bezeichnet die Mathmatische Definition welche von Imaginären Zahlen abgeleitet ist.

\begin{gather*}
 d = x + \sum_{i = 1}^{k} y_i \epsilon_i \\
	\text{ d ist eine Multidimensionale Dual Zahl } \\
	\text{ x ist die relle componente } \\
	y_i \text{ sind die imaginären komponenten mit } i \in \{1, ..., k\}
\end{gather*}

Bei Multidimensinalen Dual zahlen git:
$$
 \epsilon_i \cdot \epsilon_j = 0
$$

Für die Definition von MultidimensinalenDual-Zahlen ist ebenfalls ihr verhalten auf scalare Operationen wichtig:

$$
 f( x + \sum_{i = 1}^{k} y_i \epsilon_i  ) = f(x) +  f'(x) \sum_{i = 1}^{k} y_i \epsilon_i
$$
\cite{juliaForwardDiffPackage}


Das verhalten auf scalare operationen ist entscheidet für die Berechnung von durch Autodifferentiation Algorithemen.
Um die scatter-function umzusetzen, wird die Definition der Addition von MultidimensionalenDualZahlen \cite{RecentAdvances} benötigt.

$$
(x_1 + \sum_{i = 1}^{k} y_{1,k} \epsilon_i) + ( x_2 + \sum_{i = 2}^{k} y_{2,k} \epsilon_i) = (x_1 + x_2) + \sum_{i = 1}^{k} (y_{1, i}) \epsilon_i
$$

Alle weitern operation lassen sich ebenfalls von den von den Komplexen Zahlen ableiten, 
sind aber nicht relevant um die Differentialgleichungs löser zu unterstützen.

