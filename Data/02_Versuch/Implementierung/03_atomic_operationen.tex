\subsection{Atomic Operationen} \label{atomic}

Da nun bekannt ist, was eine Dual Zahl ist, 
kann das eigentliche Problem betrachtet werden.

Die Implementierung der \textit{scatter!} function, benutzt folgenden Programmcode um die 
Addition umzusetzen. 

\begin{lstlisting}{language=Julia}
	CUDA.@atomic dst[idx[index]...] = op(dst[idx[index]...], src[index])
\end{lstlisting}


Dabei ist der Macro \textit{@atomic} ein Wrapper um die C-Implementierung von CUDA.

Das Probelem das auftritt ist, dass der Macro nur die Datentypen \{Float16, Float32, Float64, BFloat16 \}
unterstützt.

Es ist möglich die unterstützten datentypen zu erweitern. 
Vorraussetzung dafür ist, das der neue datentype kleiner gleich 64-Bit ist.
Da eine Dual Zahl nur für $k = 1$ und Datentype \textit{Float32} genau 64-Bit groß ist,
kann nur in besondren fällen \textit{@atomic} direkt verwendet werden.
Da dieser Sonderfall, meistens nicht auftritt ist dieser ansatzt unbrauchbar.

Der \textit{@atomic} Aufruf wird benötigt, 
um \textit{Read-Modify-Write} operationen zu Serialisieren, 
und \textit{Race-Conditions} zuvermeiden und kann deshalb auch nicht weggelassen werden.

Aus der Definition der Multidimensinalen Dual Zahlen, ist bekannt das \textit{dual}-Zahlen addiert werden 
indem die komponenten addiert werden.

Da die einzelnen komponenten Rellen zahlen, 
bzw. in der Praxis Floats sind, liegt folgende implementierung nahe:

\begin{lstlisting}{language=Julia}

function gpu_add!(d0::Dual, d1::Dual, d2::Dual)
	@atomic value(d0) = value(d1) + value(d2)
	
	for i in 1:length(partials(d0)) # length(partials( ... )) ist für alle Argumente Gleich
		@atomic partials(d0, i) = partials(d1, i) + partials(d2, i)
	end
end

\end{lstlisting}

Wird die scatter Funktion nun wie beschrieben implementiert, kommt es zu einem weiteren Problem.
Der Macro @atomic verbietet die Funktions aufrufe \textit{value}und \textit{partial}.
Dies liegt daran, das ein Macro in der Julia sprache zugang auf den \textit{Abstract Syntax Tree} des Complieler hat
und einen direkten zugriff auf ein \textit{Array} erwarted.
Der Aufruf der partial-Funktion führt außerdem skalare Indexierung durch, welche auf der Grafikkarte nicht erlaubt ist.

