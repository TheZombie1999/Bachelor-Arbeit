% TODO
% Machen sätzte sinn
% Quellen
% Englische Worte Kursiv
% Rechtschreibung und Gramatik

\section{Ziel der Arbeit}

Der Zweck dieser Arbeit ist es, Nutzern von \textit{NeuralODEs} es zu ermöglichen, 
verschiedene Klassen von DifferentzialGleichungs-Lösern zu nutzen.


Da explizite Löser ohne zusätzliche arbeit nutztbar, 
sind konzentiert sich diese Arbeit sich auf Implizite Löser. 
Die größte Herausforderung dabei ist , 
diese auf der Graphikkarte und nicht auf der CPU zu berechnen.
Der Grund ist die lange berechnungs dauer der impliziten Methoden.
Das Hauptproblem beim berechnen von \textit{NeuralODEs} ist, 
dass die impliziten Methoden als zeit eingabe nicht immer relle Zahlen verwenden,
sondern \textit{Dual}-Zahlen, welche eine abwandlung von imaginären Zahlen sind.
Das Verwenden von Dual Zahlen ist voraussetzung dafür, 
um automatische Differenzierungs algorithmen berechenen zu können.
Dies setzt vorraus dass alle komponenten der \textit{NeuralODE} diese verarbeiten können.
Die Meisten komponenten sind dazu bereits in der lage, 
da die implementierung der \textit{Dual}-Zahlen
die standart operationen überlädt.
Auf der Grafikkarte tritt das Problem auf das die standart operationen, 
zusätzlichen beschränkungen unterliegen, 
und deshalb \textit{Dual} Zahlen nicht verarbeiten können.
In diesem Kaptiel werden die hintergründe erläutert und ein Lösungsansatz vorgestellt.


