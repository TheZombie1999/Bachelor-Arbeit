
% Inhalt

% \chapter{Versuch}
% \section{Ziel der Arbeit}
% \section{Implementierung der Scatter funktion}
% \subsection{Vorstellung der Scatter funktion}
% \subsection{Dual Zahlen}
% \subsection{atomic}
% \subsection{Eine Funktionierende Implementierung}
% \subsection{Test}
% \section{ Fehler behebung in den Impliziten Lösern }



\chapter{Versuch}

% \section{Ziel der Arbeit}

% Der Zweck dieser Arbeit ist es, Nutzern von \textit{NeuralODEs} es zu ermöglichen, 
% verschiedene Klassen von DifferentzialGleichungs-Lösern zu nutzen.
% Da explizite Löser bereits nutztbar, 
% sind konzentiert diese Arbeit sich auf Implizite Löser. 
% Die größte Herausforderung ist dabei, 
% diese auf der Graphikkarte und nicht auf der CPU zu berechnen.
% Der Grund ist die lange berechnungs dauer der impliziten Methoden.
% Das Hauptproblem beim berechnen von \textit{NeuralODEs} ist, 
% dass die impliziten Methoden als zeit eingabe nicht immer relle Zahlen verwenden,
% sondern \textit{Dual} Zahlen, welche eine abwandlung von imaginären Zahlen sind.
% Das Verwenden von Dual Zahlen ist voraussetzung dafür, 
% um automatische Differenzierungs algorithmen berechenen zu können.
% Dies setzt vorraus dass alle komponenten der \textit{NeuralODE} diese verarbeiten können.
% Die Meisten komponenten sind dazu bereits in der lage, 
% da die implementierung der \textit{Dual} Zahlen
% die standart operationen überlädt.
% Auf der Grafikkarte tritt das Problem auf das die standart operationen, 
% zusätzlichen beschränkungen unterliegen, 
% und deshalb \textit{Dual} Zahlen nicht verarbeiten können.
% In diesem Kaptiel werden die hintergründe erläutert und ein Lösungsansatz vorgestellt.


\section{Implementierung der Scatter funktion}

% \subsection{Vorstellung der Scatter funktion}

% Um die Probleme der \textit{scatter} funktion lösen zukönnen, 
% wird dessen funktionsweiße vorgestellt.
% Die Aufgabe der \textit{scatter} function ist es werte aus einem N-Dimensinalen Array in ein anders 
% ähnlicher Größe Array zuschreiben und dabei auf alle in eine zelle geschriebenen werte eine Operation anzuwenden.
% Es können zum beispiel mehrere eingaben in einen Knoten eines Graphen, 
% welcher in Matrix form aufgestellt wurde,
% zusammen gerechnet werden.
% Die besonderheit der scatter function ist dabei, 
% das dies nicht nur für eine Zelle geschieht, 
% sonder für alle.
% Die \textit{scatter} function ist wie folgt definiert:

% \begin{lstlisting}{language=Julia}
% 	"""
% 	op ist die operation die auf alle werte die in die selbe zelle geschrieben werden angewendet wird.
% 		mögliche werte sind +, -, *, /, min, max, mean
% 	src ist eine beliebige matrix aus der wert gelesen und wieder an anderer stelle zurück geschrieben werden.
% 	idx ist ebenfalls eine beliebige matrix aus der entnommen wird an welche position, der wert einer zelle geschrieben werden soll.
% 	init beschreibt wie das ziel array initialisiert wird.
% 			dies kann zum beispiel genutzt werden, um den inhalt des arrays in das geschrieben wird initial auf null zu setzten.
% 	dstsize gibt die größe des zurück gegebenen arrays an.
% 	"""

% 	function scatter(op, src, idx; [init, dstsize])
% 		dst = similar(src)

		
% 		fill!(dst, init)
		
% 		scatter!(op, dst, src, idx)
% 	end
% \end{lstlisting}


% Die implementierung der \textit{scatter} funktion ist eine vereinfachte darstellung, 
% welche details zur berechnung der größe des dst array weg lässt.
% Die gesamte implementierung kann in \scatter{NNlib.jl} \cite{} geschlagen werden.


% Wichtig an der Implementierung ist, dass die \textit{similar} funktion verwendet wird um, das
% \textit{dst} Array zu erstellen.

% Dieser Aufruf stellt sicher das \textit{dst} und \textit{src} den selben datentyp haben.

% Dies hat den Großen vorteil, 
% dass die beiden Arguemente der \textit{op} Operation von Gleichen Datentyp sind. 
% Dies verhindert, 
% dass viele spezialfälle der addition von Dualzahlen, nicht implementiert werden müssen.

% Es fällt ausserdem auf das die Implementierung der berechung innheralb der \textit{scatter!}
% durchgeführt wird.


% Das Ausrufe zeichen im Funktions namen, bedeutet das die argumente der Funktion überschrieben werden können.


% Die \textit{NNlibCuda.jl} Bibliothek überlädt diese Funktion für \textit{CUDA}-Arrays,
% implementiert die berechnung auf der Graphikkarte.

% Innerhalb dieser implementierung tritt der Fehler auf der die Verwendung von \textit{Dual}-Zahlen,
% verhindert.
	
% \subsection{Dual Zahlen}

% Um zu verstehen, was auf der Graphikkarte zu Problemen führt, wird zunächst die Definition von \textit{Dual}-Zahlen betrachtet.

% Da sich hinter diesem Begriff zwei sehr ähnliche Konzepte verbergen führen wir folgende zwei begriffe ein.

% Mit einer \textit{Dual}-Zahl wird, die konkrete implementierung in der \textit{ForwardDiff.jl} Bibliothek bezeichnet.

% Eine Multidimensionale Dual zahl bezeichnet die Mathmatische Definition welche von Imaginären Zahlen abgeleitet ist.

% \begin{gather*}
%  d = x + \sum_{i = 1}^{k} y_i \epsilon_i \\
% 	\text{ d ist eine Multidimensionale Dual Zahl } \\
% 	\text{ x ist die relle componente } \\
% 	y_i \text{ sind die imaginären komponenten mit } i \in \{1, ..., k\}
% \end{gather*}

% Bei Multidimensinalen Dual zahlen git:
% $$
%  \epsilon_i \cdot \epsilon_j = 0
% $$

% Für die Definition von MultidimensinalenDual-Zahlen ist ebenfalls ihr verhalten auf scalare Operationen wichtig:

% $$
%  f( x + \sum_{i = 1}^{k} y_i \epsilon_i  ) = f(x) +  f'(x) \sum_{i = 1}^{k} y_i \epsilon_i
% $$
% \cite{juliaForwardDiffPackage}


% Das verhalten auf scalare operationen ist entscheidet für die Berechnung von durch Autodifferentiation Algorithemen.
% Um die scatter-function umzusetzen, wird die Definition der Addition von MultidimensionalenDualZahlen \cite{RecentAdvances} benötigt.


$$
(x_1 + \sum_{i = 1}^{k} y_{1,k} \epsilon_i) + ( x_2 + \sum_{i = 2}^{k} y_{2,k} \epsilon_i) = (x_1 + x_2) + \sum_{i = 1}^{k} (y_{1, i} + y_{2, i}) \epsilon_i
$$


% Alle weitern operation lassen sich ebenfalls von den von den Komplexen Zahlen ableiten, 
% sind aber nicht relevant um die Differentialgleichungs löser zu unterstützen.


% \subsection{Atomic Operationen} \label{atomic}

% Da nun bekannt ist, was eine Dual Zahl ist, 
% kann das eigentliche Problem betrachtet werden.

% Die Implementierung der \textit{scatter!} function, benutzt folgenden Programmcode um die 
% Addition umzusetzen. 


% \begin{lstlisting}{language=Julia}
% 	CUDA.@atomic dst[idx[index]...] = op(dst[idx[index]...], src[index])
% \end{lstlisting}



% Dabei ist der Macro \textit{@atomic} ein Wrapper um die C-Implementierung von CUDA.

% Das Probelem das auftritt ist, dass der Macro nur die Datentypen \{Float16, Float32, Float64, BFloat16 \}
% unterstützt.

% Es ist möglich die unterstützten datentypen zu erweitern. 
% Vorraussetzung dafür ist, das der neue datentype kleiner gleich 64-Bit ist.
% Da eine Dual Zahl nur für $k = 1$ und Datentype \textit{Float32} genau 64-Bit groß ist,
% kann nur in besondren fällen \textit{@atomic} direkt verwendet werden.
% Da dieser Sonderfall, meistens nicht auftritt ist dieser ansatzt unbrauchbar.

% Der \textit{@atomic} Aufruf wird benötigt, 
% um \textit{Read-Modify-Write} operationen zu Serialisieren, 
% und \textit{Race-Conditions} zuvermeiden und kann deshalb auch nicht weggelassen werden.

% Aus der Definition der Multidimensinalen Dual Zahlen, ist bekannt das \textit{dual}-Zahlen addiert werden 
% indem die komponenten addiert werden.

% Da die einzelnen komponenten Rellen zahlen, 
% bzw. in der Praxis Floats sind, liegt folgende implementierung nahe:

% \begin{lstlisting}{language=Julia}

% function gpu_add!(d0::Dual, d1::Dual, d2::Dual)
% 	@atomic value(d0) = value(d1) + value(d2)
	
% 	for i in 1:length(partials(d0)) # length(partials( ... )) ist für alle Argumente Gleich
% 		@atomic partials(d0, i) = partials(d1, i) + partials(d2, i)
% 	end
% end

% \end{lstlisting}


% Wird die scatter Funktion nun wie beschrieben implementiert, kommt es zu einem weiteren Problem.
% Der Macro @atomic verbietet die Funktions aufrufe \textit{value}und \textit{partial}.
% Dies liegt daran, das ein Macro in der Julia sprache zugang auf den \textit{Abstract Syntax Tree} des Complieler hat
% und einen direkten zugriff auf ein \textit{Array} erwarted.
% Der Aufruf der partial-Funktion führt außerdem skalare Indexierung durch, welche auf der Grafikkarte nicht erlaubt ist.



% \subsection{Memory Allocation}

% Wie bereits im Kaptiel \ref{atomic} beschrieben, muss die addition von Dual zahlen komponenten weiße durchgeführt werden.
% Da es aber nicht möglich ist auf die einezelen komponentn direkt zuzugreifen müssen diese zwischen gespeichert werden um direkten zugriff zu erlauben.
% Der erste ansatzt, dazu ist speicher direkt im kernel zu reservieren.
% Dies scheitert daran, dass andere Threads auf diesen speicher nicht zugreifen können.
% Um speicher zu erstellen der von allen threads verwendet werden kann, kann shared memory verwendent werden.
% Das problem dabei ist das shared memory nur von threads im gleichen thread block sind zugegriffen werden kann.
% Es kann global speicher reserviert werden, indem dieser Ausserhalb des Kernels allociert und dann beim start diesem als arguemnt übergeben wird.
% Dieser Ansatz führt zu erfolg, indem ein eigenenr speicher bereich für jede komponente einer dual zahl reserviert wird, den entsperechenden wert in diesen hinein copiert
% und dann die addition analog zu \ref{} innerhalb der \textit{scatter!} function implementiert.
% Dieser Ansatz hat zwei probleme.
% Das erste Problem ist, dass große teil der bereits vorhandenen implementierungen der \textit{scatter!} function mit minimalen änderungen reimplementiert werden müssen, was 
% gegen das dry-Prizip verstößt.
% Zweitens ist es aktuell zwar möglich die berechnung korrekt durchzuführen, dass ergebniss kann aber noch nicht in das ziel array zurück geschrieben werden.
% Das nächste Kaptiel bespricht wie genau diese zwei Probleme gelößt werden können.


% \subsection{Eine Funktionierende Implementierung}

% In diesem Kaptipel geht es darum, wie die \textit{scatter} function nun umgesetzt werden kann.

% Dazu werde die argumente so umgeformt, 
% dass die bereits vorhandenen implementierungen der \textit{scatter!} function, diese richtig verarbeiten können.

% Es wird dafür zurerst CUDA-speicher reserviert und dann werden Kernels verwenden um 
% die entsprechenede werte in diesen zu kopieren.

% Ein kernel ist eine Function die Auf der Graphikkarte ausgeführt wird.

% Wichtig ist dabei das der CUDA-speicher die richtige Dimension haben. 

% Das ursprünglichen Arrays \textit{src} und \textit{dst} haben die Dimension N.

% Das Array \textit{idx} entweder dimension N und enthält tupel von gleicher Größe,
% oder dimension 1 und und enthält \textit{Interger} Zahlen.

% Falls die Dimension von \textit{idx} Eins ist, und die Dimension von \textit{src}
% und \textit{dst} größer dann wird \textit{idx} zeilen weiße angewendet.

% Es wird nun eine Kopie dieser Drei Arrays erstellt mit Dimension N+1.

% In der Extra Dimension werden die einzelnen Komponenten der Dual Zahl gespeichert.

% Ein Beispiel für $N = 1$, sieht für \textit{src} und \textit{dst} sieht wie folgt aus:


$$
\text{src oder dst} = 
\begin{matrix}
Dual & Dual & Dual
\end{matrix} \\
\rightarrow\\
\text{src\_m oder dst\_m} = 
\begin{matrix}
Value   & Value   & Value   \\
Partial & Partial & Partial \\
Partial & Partial & Partial \\
Partial & Partial & Partial \\
\end{matrix}
$$

% Das \textit{idx} Array ist in dieser hinsicht besonders.
% Da die einzelnen \textit{Integer}-Zahlen zu Tupel um gewandelt werden müssen. 


$$
\text{idx}= 
\begin{matrix}
 2 & 1 & 3
\end{matrix}
\rightarrow
idx\_m = 
\begin{matrix}
 (2, 1) & (1, 1) & (3, 1) \\ 
 (2, 2) & (1, 2) & (3, 2) \\
 (2, 3) & (1, 3) & (3, 3) \\
 (2, 4) & (1, 4) & (3, 4) \\
\end{matrix}
$$


% Die Anzahl der Zeilen in beiden Beispielen, ist abhänig von der anzahl der Freien variablen  
% der Multidimensionalen Dual Zahlen.

Wird nun \textit{dst\_m}, \textit{src\_m} und \textit{idx\_m} an die \textit{scatter!} function über
geben, wird das erwünschte ergebnis berechnet. 


% Es muss nur noch in das korekte format zurück kopiert werden. 

% Für diesen Letzten schritt müssen mehrere \textit{views} auf
% \textit{dst_m} erstellt werden.

% Das verwenden von \textit{views} Löst das Problem von scalarer indezierung innerhalb der Graphikkarten Kernel. 

% Eine \textit{view} erlaubt es die art und weiße wie auf einen speicher bereich zugegriffen wird zu verändern.

% Diese werden auf der CPU erstellt und dann dem Kernel als argument übergeben.

% Um zum Beispiel ein Array von Tupeln, mit den Partials als inhalt zuerhalten, wird folgender Code ausgeführt:

\begin{lstlisting}
	function f(x...)
		return x
	end
	
	tuples = f.([view(dst_i, repeat([:], 
	ndims(dst_i) - 1)..., i) for i in 2:(N+1)]...)
\end{lstlisting}



% Für mehre details zur genauen umsetzung, kann der Code im anhang eingesehen werden.


% \subsection{Testen}

% Der letzt schritt ist nun die zuvor beschriebene implementierung zu testen.

% Um die Implementierung der \textit{scatter} funktion zu testen, wird ausgenutzt,
% das es eine funktionierende CPU implementierung von der \textit{nnlib.jl} Bibliothek gibt.


Dies hilft ausserdem im Nächsten Kapitel die Performance der Grafikkarten berechnung zu evaluieren.

% Es wird also ein CPU und CUDA-Array erstellt und dann das ergebnis verglichen.


% Dies hilft ausserdem im Nächsten Kapitel die Performance der Grafikkarten berechnung zu evaluieren hilft im Nächsten Kapitel die Performance der Grafikkarten berechnung zu evaluieren.

% \section{ Fehler behebung in den Impliziten Lösern }

% Da nun die scatter Funktion mit \textit{dual} zahlen auf der GPU durchgeführt werden kann, 
% können prinzipiel implizite Differentialgleichungs löser für 
% NeuralODEs auf der Graphikkarte verwendet werden.

% Julia hat sich im laufe der zeit weiter entwickelt und einig API Änderungen vorgenommen.

% Dies hat auf der Seite der Löser dazu geführt, dass diese zum Teil Funktionalität von Julia 1.8 wollten 
% und zum Teil Funktionalität von den früheren Versionen.
% Das Problem war das sich die Implementierung der lu-zerlegung geändert hat. % lu-zerlegung
% In den älteren Versionen hat CUDA diese Implementierung selbst bereitgestellt.
% Mit der Version 1.8 wurde dies allerdings in die Standard Bibliothek mit aufgenommen, das die Faktorisierung 
% auch auf der Grafik karte durchgeführt werden kann.
% Das OrdenaryDifferentialEquations packet von Julia verwendet 
% ein Array-Interface, welches operationen auf Arrays, 
% von der Umsetzung abstrahiert, um diese auf einer Grafikkarte durchführen zu können.
% Dank der Hilfe von Christopher Rackauckas ist es, dann nach vielen versuchen gelungen, die lu\_instance Funktion, von \textit{ArrayInterface.jl} richtig umzusetzen 
% und es damit möglich zu machen zumindest die im nächsten Kapitel vorgestellten impliziten Löser zu verwenden.


