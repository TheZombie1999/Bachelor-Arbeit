

\section{Anfangswert Probleme}
% Numerical Integraion book
% https://books.google.de/books?hl=de&lr=&id=gGCKdqka0HAC&oi=fnd&pg=PP1&dq=numerical+integration&ots=NDzzuAvLhM&sig=3_fDwVwrMX_kgecuXDtsKIZ6pgE&redir_esc=y#v=onepage&q&f=false

% https://en.wikipedia.org/wiki/Numerical_integration#:~:text=In%20analysis%2C%20numerical%20integration%20comprises,numerical%20solution%20of%20differential%20equations.

% Anfangswert Probleme

Ein Anfangswertproblem 1. Ordnung ist, ein Gleichungssystem das wie folgt definiert ist:
$$
\frac{dy}{dt} = f(t, y(t)) \text{ mit der anfangs bedingung } y(t_0) = y_0
$$

Anfangs wert Probleme spielen eine Große Rolle in der Physik.

Ein bekanntes beispiel ist die beschleunigtes auto.

Beschleunigt ein auto mit konstanter beschleunigung so lautet, dass daraus entstehende anfangswertproblem:

$$
a = \frac{dv}{dt} = k \text{ wobei k const.}
$$

die Geschwindichkeit am anfang ist 0 daher gilt:

$$
v(0) = 0
$$

Dieses System hat eine Einfache exacte Lösung:
$$
v = v_0 + \int_{t_0}^{t_1} k dt = k \cdot (t_0 - t_1) 
$$

Im allgemeinen ist es aber nicht immer möglich ein Anfangswertproblem analytisch zu lösen, deshalb wurde verschiedene numersiche verfahren entwickelt um anfangswert problem numerisch zu lösen.

\section{Numerisches Lösen von Anfangswert problemen}

\subsection{Explizite Euler Verfahren}

% https://www.biancahoegel.de/mathe/verfahr/euler-verfahren_explizit.html

Das einfachste verfahren zum Lösen dieser Anfangswert probleme ist das explizite Euler verfahren.

Das explizit euler verfahren approximiert verschiedene werte des Anfangswert problem 
im abstand einer konstanten schrittweite h .

Die approximierten zeitpunkte sind wie folgt definiert.
$$
t_k = t_0 + kh \text{, mit } k = 0, 1, 2, ...
$$

Die approximierten werte von y werden dann wie folgt berechntet:

$$
y_{k + 1} = y_{k} + h \cdot f(t_k, y_k)
$$

Das Euler-Verfahren verwendet folgende approxmation:

$$
\int_{t_k}^{t_{k+1}} f(t_k, y_k(s)) ds \approx h f(t_k, y_k)
$$

Dies ist wichtig, da alle verfahren zum numerischen lösen von anfangswert problemen, sich hauptsächlich in dieser approximation unterscheiden.

Das Expliziten Euler-verfahren mit einer neural ode zu berechnen
ist sehr leicht, da $y_k$, h und $f(t_k, y_k)$ bekannt sind.

Das nächste verfahren das vorgestellt wird verändert die art und weiße der Approximation, was dazu führt das die berechnung nicht mehr so direkt durchgeführt werden kann.



\subsection{Implizite Euler Verfahren}

Das implizite Euler verfahren ist sehr ähnlich zu der expliziten variante.

Der entscheidende unterschied ist, dass das implizite verfahren nicht die steigung abhänig von aktuellen 
zeitpunkt und y-wert verwendet, sondern vom nächsten zeitpunkt.

Die Definition des lautet deshalb wie folgt.

$$
y_{k + 1} = y_k + h \cdot f(t_{k + 1}, y_{k + 1})
$$

Der rest der definition ist analog zum expliziten euler verfahren aus Kaptiel \ref{}.

Das besondere dabei ist das $f(t_{k + 1}, y_{k + 1})$ nicht bekannt ist.

Deshalb muss dieses gleichungs system gelößt werden.

Dazu wird meist das Newton verfahren verwendet.

Dazu muss die gleichung zunächst umgeformt werden.

$$
0 = (y_k - y_{k + 1})  + h \cdot f(t_{k + 1}, y_{k + 1})
$$

Nun wird die Funktion $g(x)$ wie folgt definiert:

$$
g(x) := (y_k - y_{k + 1})  + h \cdot f(t_{k + 1}, y_{k + 1}) = (y_k - x)  + h \cdot f(t_{k + 1}, x)
$$
$$
g'(x) = -1 + h \cdot \frac{\partial f(t_{k+1}, x)}{\partial x}
$$

Nun kann dessen Nullstelle über das Newton verfahren wie folgt bestimmt werden.

$$
x_{n+1} = x_{n} - \frac{g(x_n)}{g'(x_n)}
$$

Dabei ist des wichtig $x_{0}$ auf eine Sinnvollen wert zu setzen ein guter kandidat ist zum beispiel
$y_{k}$ da dieser bekannt und nahe am gesuchten wert liegt.

Da dieser nahe am gewünschten Wehrt liegt und bekannt ist.

Wichtig ist das damit das implizit euler verfahren berechnet werden kann die ableitung von $f(t_{k + 1}, y_{k + 1})$ benötigt wird.

In userem fall handelt es sich hierbei um eine NeuralODE dessen ableitung mit einem automatischen ableitungs algorithmus berechnet wird. Wie dies funktioniert wird in Kaptitle \ref{} beschrieben.



\subsection{Die Runge Kutta verfahren}

Die Runge Kutta verfahren sind eine verallgemeinerung der
bereits vorgestellten impliziten/expliziten Euler verfahren.

Die Runge Kutta verfahren verwenden folgende approximation:

$$
\int_{t_k}^{t_{k+1}} f(t_k, y_k(s)) ds \approx h \sum_{j=1}^{s} b_j k_j
$$
wobei $k_j$ wie folgt definiert ist.
$$
k_j = f(t_n + h c_j, y_n + h \sum_{l=1}^{s}a_{jl}k_{l}) , j \in \{1, ..., s\}
$$
% https://www.youtube.com/watch?v=6cqn8cnYRUg&list=PLgPpaTsP_3DqH0RNVsSOiohhGQz_7vf_R&index=7


Um Runge Kutta Methoden lösen zu können, wird im allgemeinen die Newton's Methode für ein 
system von gleichungen verwendet werden.

Dieses unterscheidet sich von den normalen Newton verfahren.

Die grund vorraussetzung für dessen berechung ist jedoch die gleiche.

Es werden die Paritellen ableitungen von der funktion 
$f$ benötigt.


\subsection{Mehr Schritt Methoden (Multistep Methods)}

Mehr schritt methoden erweitern die ursprüngliche definition des anfangswert problem darum 
das bereits eine beliegiege menge an punkten 
vor dem aktuellen zeit schritt bekannt ist.
Um den nächsten zeitschritt zuberechnen wird dann 
zwischen den bekannten werten interpoliert um den 
nächsten zeitschritt zu berechnen.

Um die initialen punkte zu berechnen wird meistens 
einen implizite Runge Kutta methode verwendet.









