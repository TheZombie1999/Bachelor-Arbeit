\chapter{Numerical Integration}

% Numerical Integraion book
% https://books.google.de/books?hl=de&lr=&id=gGCKdqka0HAC&oi=fnd&pg=PP1&dq=numerical+integration&ots=NDzzuAvLhM&sig=3_fDwVwrMX_kgecuXDtsKIZ6pgE&redir_esc=y#v=onepage&q&f=false

% https://en.wikipedia.org/wiki/Numerical_integration#:~:text=In%20analysis%2C%20numerical%20integration%20comprises,numerical%20solution%20of%20differential%20equations.

% Anfangswert Probleme

Ein Anfangswertproblem 1. Ordnung ist, ein Gleichungssystem das wie folgt definiert ist:
$$
\frac{dy}{dt} = f(t, y(t)) \text{ mit der anfangs bedingung } y(t_0) = y_0
$$

Anfangs wert Probleme spielen eine Große Rolle in der Physik.

Ein bekanntes beispiel ist die beschleunigtes auto.

Beschleunigt ein auto mit konstanter beschleunigung so lautet, dass daraus entstehende anfangswertproblem:

$$
a = \frac{dv}{dt} = k \text{ wobei k const.}
$$

die Geschwindichkeit am anfang ist 0 daher gilt:

$$
v(0) = 0
$$

Dieses System hat eine Einfache Numerische Lösung:
$$
v = v_0 + \int_{t_0}^{t_1} ...
$$

Im allgemeinen ist es aber nicht immer möglich ein Anfangswertproblem analytisch zu lösen, deshalb wurde verschiedene numersiche verfahren entwickelt um anfangswert problem numerisch zu lösen.

\section{ Explizite Euler Verfahren}

% https://www.biancahoegel.de/mathe/verfahr/euler-verfahren_explizit.html

Das einfachste verfahren zum Lösen dieser Anfangswert probleme ist das explizite Euler verfahren.

Das explizit euler verfahren approximiert verschiedene werte des Anfangswert problem 
im abstand einer konstanten schrittweite h .

Die approximierten zeitpunkte sind wie folgt definiert.
$$
t_k = t_0 + kh \text{, mit } k = 0, 1, 2, ...
$$

Die approximierten werte von y werden dann wie folgt berechntet:

$$
y_{k + 1} = y_{k} + h \cdot f(t_k, y_k)
$$

Das Euler-Verfahren verwendet folgende approxmation:

$$
\int_{}
$$






% Explicit Euler

% Implicit Eulter

% Runge Kutta


% Unterschied
