
\section{Grafikkarten Programmierung} \label{sec:gpu}

Da das Simulieren von physikalischen vorgängen, 
meistens sehr Zeitintensive berechnungen zugrunde liegen, 
wurden verschiedene Methoden entwickelt, 
um diese zu beschleuningen.
Meistens wird die CPU benutzt um diese berechnungne durch zuführen.
Die CPU ist sehr gut darin eine Sequenz von einnander abhängigen Operationen durch zuführen.
Bei der Berechung von neuronalen Netzten oder Simulationen muss aber meistens eine große Menge 
an von einander unabhängigen Operationen durch geführt werden.
Da die einzelnen Operation unabhängig voneinander sind.
Ist es irrelevant in welcher Reihenfolge diese ausgeführt werden.
Dies macht sich die Graphikkarte zu nutzen und führt eine groß Zahl an Operationen einafch gleichzeit aus.



% Mehr bezug zur eigenlichen arbeit

% mehr auf neuronale netzte eingehen.

% Matrix darstellung von neuralode zeigen

% 