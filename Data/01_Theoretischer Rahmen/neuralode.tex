\section{\textit{NeuralODEs}}

Eine NeuralODE basiert auf dem in Kaptiel \ref{} vorgestellten anfangswert problemen.

Der unterschied zwischen ein NeuralODE und einem anfangswertproblem ist, 
die Funktion f in einer NeuralODE nicht mehr eine Klassische gleichung sonder ein 
neuronales netzt repräsentiert wird.

Da ein NeuronalesNetzt nicht nur von seinen eingaben sonder auch von seien gewichten und biasies abhängt
lautet die defninitin einer NeuralODE wie folgt:

$$
\frac{d y(t)}{dt} = f(y(t), t, \theta)
$$

Durch den Trainings prozess des Neuronales Netzes werden die Gewichte und biasis des NeuronalesNetzes 
so gewählt das die funktion $f(y(t), t, \theta)$ eine Approximation der Klassischen Differentialgleichung 
$f(y(t), t)$ darstellt.

Der Grund führ diese Approximation ist, dass das NeuronaleNetz mehrere vorteile gegenüber der Klassischen DifferentialGleichung besitzt.

% Vorteil mit betreuer durch sprechen.

% https://book.sciml.ai/notes/03-Introduction_to_Scientific_Machine_Learning_through_Physics-Informed_Neural_Networks/





