
\section{Flüssigkeits simulationen} \label{sec:simulationen}

Da MeshGraphNets in der Regel verwendet werden um reale Probleme zu simiulieren, 
wird im folgendem eine Anwendung dieser genauer betrachtet.
Für die Analyse dieser Arbeit wird eine Flüssigkeits simulation verwendet.
Es gibt eine viel Zahl an verfahren um Flüssigkeiten zu simulieren.
Viele davon werden allerdings, in der Computer-Grafik verwendet um 
effizient den anschein einer Flüssigkeit zu erwecken.
Diese verfahren verwenden in der Regel eine vereinfachte version der physikalischen Gleichungen,
um diese schneller berechnen zu können.
Für die zwecke dieser Arbeit sind vorallem möglichst Realistische verfahren intressant.
Eines der ersten bekannten Verfahren zur beschreibung von Flüssigkeiten ist die Euler-Gleichung \cite[Kapitel~1]{navier_stokes}:

$$
\frac{\partial \vec{u}}{\partial t} + (\vec{u} \cdot \nabla) \vec{u} =  \vec{g} - \frac{1}{\rho} \nabla \cdot p
$$

Die Euler-Gleichung beschreibt das Verhalten von Flüssichkeiten annhand von zwei Phänomenen.
Das erste Phänomen beschreibt das verhalten, annhand von unterschiedlichen Drücken innerhalb einer Flüssigkeit.
In der Euler-Gleichung spiegelt sich dieses Phänomen in Gradienten des Druckes $p$ wieder.
Das zweite Phänomen das die Euler-Gleichung berücksichtigt ist, dass Flüssigkeiten mit
einer höhren Dichte nach unten sinken und demnach Flüssigkeiten mit einer geringen Dichte aufsteigen.
Dafür verantwortlich ist die Gewichtskraft $\vec{g}$ und Dichte $\rho$.
Die linke Seite der Gleichung beschreibt, dass sich die Geschwindichkeit $u$ der Flüssigkeit sowohl im laufe 
der Zeit, als auch abhängig vom Ort ändert.
Wird die Euler-Gleichung nun leicht umgestellt kann diesem mit einem der in Kapitel \ref{sec:numerisches_lösen_von_anfangswert_problemen} vorgestellten 
numerischen Differentialgleichungslöser berechnet werden.

$$
f(u(t), t) = \frac{\partial \vec{u}}{\partial t}  =  \vec{g} - \frac{1}{\rho} \nabla \cdot p -  (\vec{u} \cdot \nabla) \vec{u}
$$

Bei dem numerischen Lösen der Differentialgleichung kommt es zu dem problem das sich das druck Feld p
abhängig von der Geschwindichkeit ebenfalls ändert.
Diese Problem kann gelößt werden indem die Differentialgleichung in zwei teile geteilt wird.
Genauers kann in folgendem Paper nachgelesen werden \cite{num_navier}.
Die Euler-Gleichungs vernachlässicht ein entscheidendes Phähnomen.
Sie vernachlässicht das es innerhalb der Flüssichkeiten und an grenz flächen zu einer reibungskarft kommt.
Soll zum beispiel der Luftwiederstand und Auftrieb eines Tragflächen elementes bestimmt werden,
dann sind diese Reibungskraft besonder wichtig da sie für eine verlangsamung des Luftstroms nahe an der Oberfäche
des flügels sorgt für Phänomene die dessen auftrieb und Luftwiederstand beeinflussen.
Dieser Druck unterschied zwischen der ober und unter seite des Flügels sind verantwortlich für dessen auftrieb.
Wird die Euler-Gleichung nun um dieses Phänomen erweitert ergibt sich die bekannte Navier-Stokes-Gleichung \cite[Kapitel~1.4]{navier_stokes} 
welche wie folgt lautet:

$$
\rho \frac{\partial \vec{u}}{\partial t} + \rho (\vec{u} \cdot \vec{\nabla} ) \vec{u} = 
\rho \vec{g} - \nabla \cdot \vec{p} + \eta \Delta \vec{u}
$$

Der große Unterschied der Navier-Stokes-Gleichung zur Euler-Gleichung ist der Term $\eta \Delta \vec{u}$.
$\eta$ ist dabei eine Materialkonstante für die Viskosität.
Die Navier-Stokes-Gleichung ist unter keinen Umständen vollständig.
Das heißt, es gibt Phänomene, die in der Gleichung nicht berücksichtigt werden.
Oft wird dies deutlich gemacht, indem ein Kraft-Vektor $\vec{f}$ zur rechten Seite addiert wird.
Für die meisten realen Anwendung reichen diese Annahmen jedoch aus, weshalb die Navier-Stokes-Gleichung sehr nützlich ist.

Die Navier-Stokes-Gleichung ist für uns besonders interessant, da sie die Datengrundlage für das Training unserer \textit{NeuralODE} liefert.
Außerdem demonstriert sie, welche Vereinfachungen das Verwenden der NeuralODE beim Auswerten ermöglicht.
Diese erlaubt es uns, die Differenzialgleichung nur abhängig von der Geschwindigkeit $v$ auszuwerten.
Demnach kann die Berechnung des Druckes $p$ vermieden werden.

% https://www.youtube.com/watch?v=OjbQbPSopu0