
\section{Grafikkarten Programmierung} \label{sec:gpu}

Da das Simulieren von physikalischen Vorgängen 
meistens sehr zeit-intensive Berechnungen zugrunde liegen, 
wurden verschiedene Methoden entwickelt, 
um diese zu beschleunigen.
Meistens wird die CPU benutzt, um diese Berechnungen durchzuführen.
Die CPU ist sehr gut darin, eine Sequenz von einander abhängigen Operationen durchzuführen.
Bei der Berechnung von neuronalen Netzen oder Simulationen muss aber meistens eine große Menge 
an voneinander unabhängigen Operationen durchgeführt werden.
Da die einzelnen Operationen unabhängig voneinander sind.
Ist es irrelevant, in welcher Reihenfolge diese ausgeführt werden.
Dies macht sich die Grafikkarte zu nutzen und führt eine große Zahl an Operationen gleichzeitig aus.
Um zum Beispiel eine \textit{NeuralODE} auswerten zu können, muss folgende Gleichung gelöst werden:

$$
SN_i(x) = \sigma ( W_i \times x + t_i )
$$
$$
NN(x) = SN_1 \circ ... \circ SN_n (x)
$$

Wobei $\sigma$ die Aktivierungsfunktion ist, $W_i$ ist die Gewichte Matrix und $t_i$ die Bias Matrix des neuronalen Netzes.
$SN_i$ steht für eine einzelne Schicht des neuronalen Netzes und $NN$ für das gesamte Netz.
Das entscheidende dabei ist, dass für die Berechnung des neuronalen Netzes n-Mal die Addition Multiplikation und ein Broadcast der Aktivierungsfunktion
$\sigma$ durchgeführt werden muss.
Jeder dieser Operationen führt eine viel Zahl unabhängiger Operationen auf einer Matrix aus.
Diese sind der Grund, warum die Verwendung der GPU von so großer Bedeutung für das Auswerten von \textit{NeuralODEs} ist.
MeshGraphNets profitieren ebenfalls stark von der Berechnung auf der Grafikkarte.
Da die Graphen-Darstellung des Meshes eine Matrix ist.
Eine Operation, die häufig durchgeführt werden muss, ist das Weiterleiten von Informationen eines Knotens an seine Nachbarn.
Dazu werden Informationen aus einer Zelle der Knoten-Matrix in eine andere Speicherzelle derselben Matrix geschrieben.
Diese Operation kann ebenfalls gut von einer Grafikkarte durchgeführt werden.
Im Vergleich zu den typischen Matrix-Operationen ist, dass nicht jede atomare Operation komplett unabhängig voneinander ist,
da es vorkommen kann, dass mehrer Operationen gleichzeitig in dieselbe Speicherzelle schreiben.
Dies führt zu dem Hauptproblem dieser Arbeit auf welches im folgendem näher eingegangen wird.

% Mehr bezug zur eigenlichen arbeit
% mehr auf neuronale netzte eingehen.
% Matrix darstellung von neuralode zeigen
% 