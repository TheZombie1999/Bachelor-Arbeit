
\section{\textit{NeuralODEs}} \label{sec:neural_ode}

Eine \textit{NeuralODE} \cite{neuralode} basiert auf dem in Kapitel \ref{sec:anfangswert_probleme} vorgestellten Anfangswert-Problemen.
Der Unterschied zwischen ein NeuralODE und einem Anfangswert-Problem ist, 
die Funktion $f$ in einer \textit{NeuralODE} nicht mehr eine klassische Gleichung, sondern ein 
neuronales Netzt repräsentiert wird.
Da ein neuronales Netzt nicht nur von seinen Eingaben, sondern auch von seinen Gewichten und Biasies abhängt,
lautet die Definition einer \textit{NeuralODE} wie folgt:

$$
\frac{d y(t)}{dt} = f(y(t), t, \theta)
$$

Durch den trainings Prozess des neuronales Netzes werden, dessen Gewichte und Biasis 
so gewählt das die Funktion $f(y(t), t, \theta)$ eine Approximation der klassischen Differentialgleichung 
$f(y(t), t)$ darstellt.

Ziel dieser Approximation ist es, das Auswerten von großen Physik-Simulationen zu beschleunigen.
Der Grund für die erhoffte beschleunigte Auswertung ist, dass sich neuronale Netze besser parallelisiert auswerten lassen
als die klassischen Differentialgleichungen.
Ausserdem fallen Berechungen weg ,die die innere Dynamik des Anfangswert-Problems beschreiben.
So muss bei der berechnung der Navier-Stokes-Gleichung durch eine \textit{NeuralODE}, die änderung des Druck-Feldes 
nicht mit berechnet werden, da diese Dynamik von dem neuronalen Netz erlernt wurde.

Ausserdem ist es möglich die \textit{NeuralODE} auf Daten zu trainieren, 
zu denen die eigentliche Differentialgleichung nicht bekannt ist.
Soll zum Beispiel die größe einer Hasen-Population vorhergesagt werden, dann gibt es vielleicht Daten über dessen 
vergangenen verlauf, aber keine bekannte Differentialgleichung die dessen Verlauf beschreibt.
Nun kann über Finite-Differenzen-Methode die Ableitung dieser Daten berechnet werden.
Aufgrund dieser Daten wird nun die \textit{NeuralODE} trainiert.
Das trainierte neuronale Nezt kann, nun für unbekannte Zeit-Punkte ausgewertet werden.

Um diese Probleme jedoch mit einer möglichst hohen genauigkeit trainieren und auswerten zu können
sollen die in Kapitel \ref{sec:numerisches_lösen_von_anfangswert_problemen} vorgestellten Diffential-Gleichungs-Löser auf diese angewendet werden.
Um das Problem bei der Auswertung verstehen zukönnen müssen, zunächst noch \textit{MeshGraphNets} eingeführt werden.

% Der Grund führ diese Approximation ist, dass das NeuronaleNetz mehrere Vorteile gegenüber der klassischen Differentialgleichung besitzt.

% Vorteil mit betreuer durch sprechen.

% https://book.sciml.ai/notes/03-Introduction_to_Scientific_Machine_Learning_through_Physics-Informed_Neural_Networks/

% Schnellere berechnung da gewisse phänomene gelernt werden zum bsp. durck feld der navier stokes gleichung

% Daten getriebenes lernen





