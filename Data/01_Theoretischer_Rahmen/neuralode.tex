
\section{\textit{NeuralODEs}} \label{sec:neural_ode}

Eine NeuralODE \cite{neuralode} basiert auf dem in Kapitel \ref{sec:anfangswert_probleme} vorgestellten Anfangswertproblemen.
Der unterschied zwischen ein NeuralODE und einem Anfangswertproblem ist, 
die Funktion f in einer \textit{NeuralODE} nicht mehr eine klassische Gleichung sondern ein 
neuronales Netzt repräsentiert wird.
Da ein neuronales Netzt nicht nur von seinen Eingaben sondern auch von seinen Gewichten und Biasies abhängt
lautet die Defninitin einer \textit{NeuralODE}wie folgt:

$$
\frac{d y(t)}{dt} = f(y(t), t, \theta)
$$

Durch den trainings Prozess des Neuronales Netzes werden die Gewichte und Biasis des neuronalen Netzes 
so gewählt das die Funktion $f(y(t), t, \theta)$ eine Approximation der klassischen Differentialgleichung 
$f(y(t), t)$ darstellt.

Der Grund führ diese Approximation ist, dass das NeuronaleNetz mehrere Vorteile gegenüber der klassischen Differentialgleichung besitzt.

% Vorteil mit betreuer durch sprechen.

% https://book.sciml.ai/notes/03-Introduction_to_Scientific_Machine_Learning_through_Physics-Informed_Neural_Networks/

% Schnellere berechnung da gewisse phänomene gelernt werden zum bsp. durck feld der navier stokes gleichung

% Daten getriebenes lernen





