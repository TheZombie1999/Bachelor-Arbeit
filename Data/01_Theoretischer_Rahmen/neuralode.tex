
\section{\textit{NeuralODEs}} \label{sec:neural_ode}

Eine \textit{NeuralODE} \cite{neuralode} basiert auf dem in Kapitel \ref{sec:anfangswert_probleme} vorgestellten Anfangswertproblemen.
Der Unterschied zwischen einer \textit{NeuralODE} und einem Anfangswertproblem ist, 
die Funktion $f$ in einer \textit{NeuralODE} nicht durch eine klassische Gleichung, sondern durch ein 
neuronales Netz repräsentiert wird, welches nicht nur von seinen 
Eingaben, sondern auch von seinen Gewichten und Biases abhängt.
Deshalb lautet die Definition einer \textit{NeuralODE} wie folgt:

$$
\frac{d y(t)}{dt} = f(y(t), t, \theta)
$$

Durch den Trainingsprozess des neuronalen Netzes werden dessen Gewichte und Biases 
so gewählt, dass die Funktion $f(y(t), t, \theta)$ eine Approximation der klassischen 
Differenzialgleichung $f(y(t), t)$ darstellt.

Ziel dieser Approximation ist es, das Auswerten von komplexen Physik-Simulationen zu beschleunigen.
Der Grund für die beschleunigte Auswertung ist, dass sich neuronale Netze besser parallelisiert auswerten lassen
als die klassischen Differenzialgleichungen.
Außerdem fallen Berechnungen weg, die die innere Dynamik des Anfangswert-Problems beschreiben.
So muss bei der Berechnung der Navier-Stokes-Gleichung durch eine \textit{NeuralODE} die Änderung des Druck-Feldes 
nicht mit berechnet werden \ref{sec:simulationen}, da diese Dynamik von dem neuronalen Netz erlernt wird.

Außerdem ist es möglich, die \textit{NeuralODE} auf Daten zu trainieren, 
zu denen die eigentliche Differenzialgleichung nicht bekannt ist.
Soll zum Beispiel die Größe einer Hasen-Population vorhergesagt werden, über dessen vergangenen Verlauf, Daten vorhanden sind und keine 
Differenzialgleichung bekannt ist, die dessen Verlauf beschreibt,
dann kann über Finite-Differenzen-Methode die Ableitung dieser Daten berechnet werden.
Aufgrund dieser wird nun die \textit{NeuralODE} trainiert.
Das trainierte neuronale Netz kann nun für unbekannte Zeitpunkte ausgewertet werden.

Um diese Probleme jedoch mit einer möglichst hohen Genauigkeit trainieren und auswerten zu können,
sollen die in Kapitel \ref{sec:numerisches_lösen_von_anfangswert_problemen} vorgestellten Differenzialgleichungslöser auf die \textit{NeuralODE} angewendet werden.


% Der Grund führ diese Approximation ist, dass das NeuronaleNetz mehrere Vorteile gegenüber der klassischen Differentialgleichung besitzt.
% Vorteil mit betreuer durch sprechen.
% https://book.sciml.ai/notes/03-Introduction_to_Scientific_Machine_Learning_through_Physics-Informed_Neural_Networks/
% Schnellere berechnung da gewisse phänomene gelernt werden zum bsp. durck feld der navier stokes gleichung
% Daten getriebenes lernen





