

\section{\textit{MeshGraphNets}} \label{sec:meshgraphnets}

\textit{MeshGraphNets} \cite{meshgraphnets} sind eine Methode, mit der physikalische Systeme modelliert werden können.
Beispiele für solche Systeme, sind verschiedene Körper innerhalb einer Luft oder Wasser Strömung, eine 
sich aufgrund von externen Kräften verformende Platten oder eine Stoffsimulation.
Die Gemeinsamkeit all dieser Probleme ist, das zu simulierende Objekt als ein variables Mesh dargestellt wird.
Ein variables Mesh ist eine aus Dreiecken aufgebaute Gitterstruktur, welche die Simulationsdomäne räumlich diskretisiert.
Das Besondere dabei ist, dass dieses Mesh eine variable Auflösung hat, das heißt unterschiedliche Bereiche des Raumes 
haben eine unterschiedliche hohe Dichte an Punkten, durch die das Mesh dargestellt wird.
So kann in einer Flüssigkeitssimulation zum Beispiel die Grenzfläche zwischen einem Objekt und der Flüssigkeit
genauer dargestellt werden, da dort die meisten Turbulenzen auftreten.
Um so größer der Abstand von der Grenzfläche, desto laminarer verhält sich die Flüssigkeit.
Eine laminare Strömung kann durch wenige Punkte gut angenähert werden.
Der Grund für diese Vorgehensweise ist, dass dadurch die Berechnung der Simulation beschleunigt werden kann,
ohne die Genauigkeit der Berechnung zu stark zu verringern.

Die Lösung der Differentialgleichungen liefert den Zustand der Meshpunkte im Laufe der Zeit.
Diese ist nun die Grundlage für die Trainingsdaten.

Normalerweise werden nun die dem physikalischen Problem zugrunde liegenden Differenzialgleichungen mit einem numerischen Verfahren gelöst 
und damit die Attribute der Mesh-Punkte im Laufe der Zeit vorhergesagt.
Die Differenzialgleichungen für Flüssigkeits-Simulationen wird in Kapitel \ref{sec:simulationen} genauer eingegangen.


Wie diese numerisch berechnet werden, wird außerdem in Kapitel \ref{sec:numerisches_lösen_von_anfangswert_problemen} erläutert.
Mesh\-Graph\-Nets veränder nun eine entscheidende Komponente.
Sie ersetzten die klassischen Differenzialgleichungen durch ein Verfahren basierend auf neuronalen Netzen,
welche anhand von Beispieldaten, die mit den klassischen Differenzialgleichungen erzeugt werden, trainiert wird.
Da das \textit{Mesh} für jedes Problem unterschiedlich ist, muss dieses in ein einheitliches Format gebracht werden, dazu wird es in einen
Multigraphen umgewandelt.
Dieser setzt sich grundsätzlich aus drei Komponenten zusammen.
Die erste Komponente des Graphen sind alle Knoten der Simulation, diese sind analog zu den Punkten des \textit{Mesh}.
Die Kanten des MultiGraphen setzt sich aus den Welt- und Mesh-Kanten zusammen.

Welt-Kanten werden verwendet, um externe Kräfte zu modellieren und werden erzeugt, 
wenn zwei Punkte sich räumlich sehr nahe sind.
Die Mesh-Kanten sind relevant, um die Dynamiken innerhalb des \textit{Mesh} darzustellen und werden von diesem übernommen.
Über die Kanten des Multigraphen werden dann Informationen zwischen den Knoten ausgetauscht.
Diese werden dann genutzt, um die Änderung der Eigenschaften der Meshknoten zu berechnen.
Aus diesen Änderungen kann dann, mit den unterschiedlichen Differenzialgleichungslösern, der Zustand des \textit{Meshes} im nächsten Zeitschritt berechnet werden. 
Dieses Verfahren wird nun so lange wiederholt, bis der gewünschte Zeitpunkt erreicht wird.









