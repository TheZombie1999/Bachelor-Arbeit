
\section{Zusammenfassung}

In dieser Arbeit wurde gezeigt, dass implizite Differenzialgleichungslöser für Mesh\-Graph\-Nets 
unterstützt werden können, in dem die \textit{scatter}-Funktion angepasst wird.
Dabei ist die kritische Komponente, dass die Matrix von \textit{Dual}-Zahlen in eine Matrix von Floats umgewandelt wird.
Diese kann dann von der normalen \textit{scatter!}-Funktion verarbeitet werden.
Mit der Hilfe einer \textit{view} ist es dann möglich, die Matrix von Floats wieder in eine Matrix von \textit{Dual}-Zahlen
umzuwandeln.
Nachdem die impliziten Methoden nutzbar sind, wurden diese untersucht.
Das Ergebnis dieser Untersuchung ist, dass die impliziten Methoden, wie erwartet eine höhere Genauigkeit
wie die expliziten Verfahren erreichen, mit dem Nachteil, dass dessen Laufzeit deutlich länger ist.
Außerdem wurde gezeigt, dass ein guter Kompromiss zwischen den beiden Verfahren durch einen Komposit-Algorithmus erreicht werden kann.
Als Letztes wurde die GPU Version der \textit{scatter}-Funktion mit der CPU Version verglichen.
Dabei ist aufgefallen, dass die GPU Version eine ungefähr doppelte Laufzeit hat.
Dies liegt vor allem an den Speicherreservierungen, die für dessen Umsetzung benötigt wird.

