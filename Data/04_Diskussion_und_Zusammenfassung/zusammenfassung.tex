
\section{Zusammenfassung}

In dieser Arbeit wurde gezeigt, dass implizit Differenzialgleichungslöser für MeshGraphNets 
unterstützt werden kann, in dem die \textit{scatter}-Funktion um geschrieben wird.
Dabei ist die kritische Komponente, dass die Matrix von \textit{Dual}-Zahlen in eine Matrix von Floats umgewandelt wird.
Diese kann dann von der normalen \textit{scatter!}-Funktion verarbeitet werden.
Mit der Hilfe einer \textit{view} ist dann möglich, die Matrix von Floats wieder in eine Matrix von \textit{Dual}-Zahlen
umzuwandeln.
Nachdem nun die impliziten Methoden nutzbar sind, werden diese im Anschluss untersucht.
Das Ergebnis dieser Untersuchung ist, dass die impliziten Methoden, wie erwartet eine höhere Genauigkeit
wie die expliziten Verfahren erreicht, mit dem Nachteil, dass dessen Laufzeit deutlich länger ist.
Außerdem wurde gezeigt, dass ein guter Kompromiss zwischen den beiden Verfahren durch einen Komposit-Algorithmus erreicht werden kann.
Als Letztes wurde noch die GPU Version der \textit{scatter}-Funktion mit der CPU Version verglichen.
Dabei ist aufgefallen, dass die GPU Version eine ungefähr doppelte Laufzeit hat.
Dies liegt vor allem an den Speicherreservierung, die für dessen Umsetzung benötigt wird.

