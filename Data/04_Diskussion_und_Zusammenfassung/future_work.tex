
\section{Zukünftige Arbeiten}

\subsection{Mehr Schritt Methoden für ein bessres Gedächtniss}

In dem Paper von ... wurde der Vergleich zweischen NeuralODEs und RNN angesprochen.
Wird eine der Runge-Kutta-Methoden verwendet um die NeuralODE zu lösen hat grunsätzlich nur kenntniss
über den aktuellen zeit schritt.
Das heißt das neuronale Netz hat prizipiel nur ein extrem Kurzes gedächniss.
RNN haben das gleichen problem mit dem großen unterschied das diese weiter in die zunkunft schauen können, 
die sichtbarkeit der vergangen informationen, aber exponentiell abnimmt.
Es gibt mehrere ansätze diesen exponentiellen vergesssens prozess von RNNs zu umgehen, 
diese sind aber um ein vielfaches teurer zuberechnen.

Die Idee ist nun das die Mehr Schritt Methoden es dem Löser der NeuralODE ermöglichen informationen 
aus vergangenen und zukünfitigen Zeitschritten zuverbinden und dadurch der NeuralODE ein besseres 
"gedächtniss" zu geben.

Ob dies funktioniert ist eine offe Frage die noch untersucht werden kann.



