
\section{Zukünftige Arbeiten} \label{sec:future_work}

\subsection{Mehr-Schritt-Methoden für ein besseres Gedächtnis} \label{sec:better_memory}

In dem Paper von \cite{neuralode} wurde der Vergleich zwischen NeuralODEs und RNN angesprochen.
Wird eine der Runge-Kutta-Methoden verwendet, um die NeuralODE zu lösen, hat grundsätzlich nur Kenntnis
über den aktuellen Zeitschritt.
Das heißt, das neuronale Netz hat prinzipiell nur ein extrem kurzes Gedächtnis.
RNN haben das gleiche Problem, mit dem großen Unterschied, dass diese weiter in die Zukunft schauen können, 
die Sichtbarkeit der vergangenen Informationen, aber exponentiell abnimmt.
Es gibt mehrere Ansätze diesen exponentiellen vergessen Prozess von \textit{RNNs} zu umgehen, 
diese sind aber um ein Vielfaches teurer zu berechnen.
Die Idee ist nun, dass die Mehr-Schritt-Methoden es dem Löser der NeuralODE ermöglichen, Informationen 
aus vergangenen und zukünftigen Zeitschritten zu verbinden und dadurch der NeuralODE ein besseres 
„Gedächtnis“ zu geben.
Ob dies funktioniert, ist eine offene Frage, die noch untersucht werden kann.

\subsection{Laufzeitverbesserung}

Wie in Kapitel \ref{sec:isolation_of_the_scatter_funktion}
angesprochen wurde, benötigt die Speicherallokierung und 
das Hinein- und Herauskopieren der Dual-Zahlen in ein für die \textit{scatter!}-Funktion verarbeitbares Format sehr viel Zeit.
Dies ist notwendig aufgrund der Art und Weise wie das \textit{Dual}-Struct definiert ist.

\begin{lstlisting}
    struct Dual{T, V, N} <: Real
        value::V
        partials::Partials{N, V}
    end

    struct Partials{N,V} <: AbstractVector{V}
        values::NTuple{N,V}
    end
\end{lstlisting}

Da der Datentype von \textit{Partials.values} ein \textit{NTupel} ist und es sich um zwei geschachtelte 
\textit{Struct} handelt, kann auf diese mit einer \textit{view} nicht richtig zugegriffen werden.
Eine \textit{view} ist dringen notwendig, da es sonst innerhalb der \textit{scatter}-Funktion 
zu Skalaren Indizierung kommt.
Würde man die Definition der \textit{Dual}-Zahl abänder, könnte dies auf der 
GPU zu einer deutlich besseren Laufzeit führen.
Der Grund, warum die \textit{Dual}-Zahlen aktuell so definiert sind ist,
dass die CPU Version ausnutzt, das \textiti{NTupel} immutable (unveränderlich) sind und deshalb
nicht auf dem Heap, sondern auf dem Stack gespeichert werden können.
Dies führt auf der CPU zu einer bessern Performance, da weniger Speicher allokiert
werden muss.
Auf der Grafikkarte gibt es das Konzept von Stack und Heap nicht, wodurch dessen Implementierung
davon auch keinen Vorteil hat.
Das Problem dabei ist, dass wenn die Definition der \textit{Dual}-Zahl veränder wird
muss das gesamte \textit{ForwardDiff.jl} Paket und alle Bibliotheken, die diese verwenden
um geschrieben werden.
Die Laufzeit der MeshGraphNets könnte aber einen sehr großen Vorteil, daraus ziehen, weshalb 
es sich lohnen könnte sich über eine für die Grafikkarte besser geeignete Version des 
\textit{ForwardDiff.jl} Paketes Gedanken zu machen.









