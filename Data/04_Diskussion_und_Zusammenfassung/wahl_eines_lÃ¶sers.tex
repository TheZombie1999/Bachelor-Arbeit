
\section{Vor- und Nachteile der verschiedenen Differenzialgleichungslöser} \label{sec:chose_solver}
Bei der Wahl eines Löser müssen sich immer Gedanken gemacht werden, welches Ziel verfolgt wird.
Es gibt grundsätzlich 2-Kriterien für die Wahl.
Soll eine möglichst hohe Genauigkeit erzielt werden und ist die Laufzeit egal, 
dann liefern die impliziten Methoden meistens das bessere Ergebnis.
Ist vor allem eine gute Laufzeit wünschenswert, dann sind die expliziten Verfahren am besten 
geeignet.
Durch die Kombinierung der beiden Verfahren durch einen Komposit-Algorithmus kann auch eine Kombination
der beiden Vorteile erreicht werden.
Das heißt eine mittlere Laufzeit bei akzeptable Genauigkeiten.
Die Gewichtung der einzelnen Vor- und Nachteile hängt dann stark von den konkreten Einstellungen 
im Löser ab.
Ein drittes Kriterium, das manchmal interessant ist, ist der speicher verbrauch des Programms.
Die Mehr-Schritt-Methoden sind hierbei besonders interessant, da diese nur bereits berechnete Werte
die für das Endergebnis ohnehin abgespeichert werden müssen, verwenden, um eine höhere Genauigkeit zu erzielen.
Andere Verfahren werten oft die Differenzialgleichung an einer Stelle aus, die für die Ausgabe am 
Ende nicht relevant ist. Diese werden oft in einem Cache abgespeichert, da zukünftige Schritte 
diese noch benötigen könnten.



