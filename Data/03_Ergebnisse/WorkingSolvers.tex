\section{Funktionierende Löser}

In diesem Abschnitt wird ein überblick darüber gegeben, welche 
ODE Löser nach dem hinzufügern der in \ref{} Vorgestellten 
version der \textit{scatter}-Funktion nun möglich sind.

Wichtig ist dabei das es nicht möglich ist jeden löser ausführlich zu testen
da diese zum Teil eine sehr lange laufzeit haben.

Aufgrund dessen wurde jeder der im folgenden grafik als Funktionend 
bezeichnet solange laufen gelassen bis dieser den ersten zeit schritt gemacht hat.

Der Grund für diese vorgehens weiße ist dass die löser an diesem punkt meistens scheitern.

Wenn dieser punkt überschritten war dann kam es erfahrungs gemäß zu keinen 
weitern problem eine 100\%-tige garantie lieftert dies aber nicht.

\begin{table}[]
    \centering

    \begin{tabular}{c|c|c}
        Kategorie & Name & Funktioniert \\
        \hline\hline
         SDIRK Methods & Implizit Euler & ja \\
         & ImplicitMidpoint & ja\\
         & Trapezoid & ja \\
         & TRBDF2 & ja \\
         & SDIRK2 & ja \\
         & Kvaerno3 & ja \\
         & KenCarp3 & ja \\
         & Cash4 & ja \\
         & Hairer4 & ja \\
         & Hairer42 & ja \\
         & Kvaerno4 & ja \\
         & KenCarp4 & ja \\
         & KenCarp47 & ja \\
         & Kvaerno5 & ja \\
         & KenCarp5 & ja \\
         & KenCarp58 & ja \\
         Fully-Implicit Runge-Kutta & & \\
         & & \\
         & & \\
         & & \\
         & & \\
         & & \\
         & & \\
         & & \\
    \end{tabular}
    \caption{Caption}
    \label{tab:my_label}
\end{table}

