\section{Funktionierende Löser}

In diesem Abschnitt wird ein überblick darüber gegeben, welche 
ODE Löser nach dem hinzufügern der in \ref{} Vorgestellten 
version der \textit{scatter}-Funktion nun möglich sind.
Wichtig ist dabei das es nicht möglich ist jeden löser ausführlich zu testen
da diese zum Teil eine sehr lange laufzeit haben.
Aufgrund dessen wurde jeder der im folgenden grafik als Funktionend 
bezeichnet solange laufen gelassen bis dieser den ersten zeit schritt gemacht hat.
Der Grund für diese vorgehens weiße ist dass die löser an diesem punkt meistens scheitern.
Wenn dieser punkt überschritten war dann kam es erfahrungs gemäß zu keinen 
weitern problem eine 100\%-tige garantie lieftert dies aber nicht.

% \begin{table}[]
%     \centering
    
%     \begin{tabular}{p{5cm}|c|p{5cm}}
%         Name             & Funktionale & Fehler \\
%         \hline\hline        
%         ImplicitEuler    & true        & \\ 
%         ImplicitMidpoint & true        & \\ 
%         Trapezoid        & true        & \\ 
%         TRBDF2           & true        & \\ 
%         SDIRK2           & true        & \\ 
%         Kvaerno3         & true        & \\ 
%         KenCarp3         & true        & \\ 
%         Cash4            & true        & \\ 
%         Hairer4          & true        & \\ 
%         Hairer42         & true        & \\ 
%         Kvaerno4         & true        & \\ 
%         KenCarp4         & true        & \\ 
%         KenCarp47        & true        & \\ 
%         Kvaerno5         & true        & \\ 
%         KenCarp5         & true        & \\ 
%         KenCarp58        & true        & \\       
%     \end{tabular}
%     \caption{SDIRK Methods}
%     \label{tab:my_label}
% \end{table}

% \begin{table}[]
%     \centering

%     \begin{tabular}{p{5cm}|c|p{5cm}}
%         Name & Funktionale & Fehler \\
%         \hline\hline
%         RadauIIA3 & false & Error: DimensionMismatch: arguments must have the same number of rows \\
%         RadauIIA5 & false & Error: DimensionMismatch: arguments must have the same number of rows \\
%     \end{tabular}
%     \caption{Fully-Implicit Runge-Kutta Methods}
%     \label{tab:my_label}
% \end{table}

% \begin{table}[]
%     \centering

%     \begin{tabular}{p{5cm}|c|p{5cm}}
%         Name & Funktionale & Fehler \\
%         \hline\hline
%         PDIRK44 & false & Error: `@threads :static` cannot be used concurrently or nested \\
%     \end{tabular}
%     \caption{Parallel Diagonally Implicit Runge-Kutta Methods}
%     \label{tab:my_label}
% \end{table}

% \begin{table}[]
%     \centering

%     \begin{tabular}{p{5cm}|c|p{5cm}}
%         Name & Funktionale & Fehler \\
%         \hline\hline
%         ROS3P     & true & \\ 
%         Rodas3    & true & \\ 
%         RosShamp4 & true & \\ 
%         Veldd4    & true & \\ 
%         Velds4    & true & \\ 
%         GRK4T     & true & \\ 
%         GRK4A     & true & \\ 
%         Ros4LStab & true & \\ 
%         Rodas4    & true & \\ 
%         Rodas42   & true & \\ 
%         Rodas4P   & true & \\ 
%         Rodas4P2  & true & \\ 
%         Rodas5    & true & \\ 
%     \end{tabular}
%     \caption{Rosenbrock Methods}
%     \label{tab:my_label}
% \end{table}

% \begin{table}[]
%     \centering

%     \begin{tabular}{p{5cm}|c|p{5cm}}
%         Name & Funktionale & Fehler \\
%         \hline\hline
%         Rosenbrock23     & true & \\                                                                                                                          ║
%         Rosenbrock32     & true & \\                                                                                                                          ║
%         RosenbrockW6S4OS & true & \\                                                                                                                      ║
%         ROS34PW1a        & true & \\                                                                                                                             ║
%         ROS34PW1b        & true & \\                                                                                                                             ║
%         ROS34PW2         & true & \\                                                                                                                              ║
%         ROS34PW3         & true & \\            
%     \end{tabular}
%     \caption{Rosenbrock-W Methods}
%     \label{tab:my_label}
% \end{table}

% \begin{table}[]
%     \centering

%     \begin{tabular}{p{5cm}|c|p{5cm}}
%         Name & Funktionale & Fehler \\
%         \hline\hline
%             ImplicitEulerExtrapolation & true & \\ 
%             ImplicitDeuflhardExtrapolation & true & \\ 
%             ImplicitHairerWannerExtrapolation & true & \\ 
%     \end{tabular}
%     \caption{Parallelized Implicit Extrapolation Methods}
%     \label{tab:my_label}
% \end{table}

% \begin{table}[]
%     \centering

%     \begin{tabular}{p{5cm}|c|p{5cm}}
%         Name & Funktionale & Fehler \\
%         \hline\hline
%         PDIRK44 & false & Error: `@threads :static` cannot be used concurrently or nested \\     
%     \end{tabular}
%     \caption{Parallelized DIRK Methods}
%     \label{tab:my_label}
% \end{table}

% \begin{table}[]
%     \centering

%     \begin{tabular}{p{5cm}|c|p{5cm}}
%         Name & Funktionale & Fehler \\
%         \hline\hline
%         LawsonEuler  & false & Error: ArgumeError: Caching can only be used with SplitFunction \\
%         NorsettEuler & false & Error: ArgumeError: Caching can only be used with SplitFunction \\
%         ETD2         & false & Error: type ODEFunction has no field f1 \\
%         ETDRK2       & false & Error: ArgumeError: Caching can only be used with SplitFunction \\
%         ETDRK3       & false & Error: ArgumeError: Caching can only be used with SplitFunction \\
%         ETDRK4       & false & Error: ArgumeError: Caching can only be used with SplitFunction \\
%         HochOst4     & false & Error: ArgumeError: Caching can only be used with SplitFunction \\
%     \end{tabular}
%     \caption{Exponential Runge-Kutta Methods}
%     \label{tab:my_label}
% \end{table}

% \begin{table}[]
%     \centering

%     \begin{tabular}{p{5cm}|c|p{5cm}}
%         Name & Funktionale & Fehler \\
%         \hline\hline
%         Exp4      & false & Error: AssertiError: Dimension mismatch \\
%         EPIRK4s3A & false & Error: AssertiError: Dimension mismatch \\
%         EPIRK4s3B & false & Error: AssertiError: Dimension mismatch \\
%         EPIRK5P1  & false & Error: AssertiError: Dimension mismatch \\
%         EPIRK5P2  & false & Error: AssertiError: Dimension mismatch \\
%         EPIRK5s3  & false & Error: DimensionMismatch: tried to assign 2×1896 array to 3792×1 destination \\
%         EXPRB53s3 & false & Error: AssertiError: Dimension mismatch \\
%     \end{tabular}
%     \caption{Exponential Propagation Iterative Runge-Kutta Methods}
%     \label{tab:my_label}
% \end{table}

% \begin{table}[]
%     \centering

%     \begin{tabular}{p{5cm}|c|p{5cm}}
%         Name & Funktionale & Fehler \\
%         \hline\hline
%         Exprb32 & false & Error: DimensionMismatch: array could not be broadcast to match destination \\
%         Exprb43 & false & Error: DimensionMismatch: A has dimensions (3792,3792) but B has dimensions (2,1896) \\       
%     \end{tabular}
%     \caption{Adative Exponential Rosenbrock Methods}
%     \label{tab:my_label}
% \end{table}

% \begin{table}[]
%     \centering

%     \begin{tabular}{p{5cm}|c|p{5cm}}
%         Name & Funktionale & Fehler \\
%         \hline\hline
%         QNDF1 & true   & \\
%         QBDF1 & true   & \\
%         ABDF2 & true   & \\
%         QNDF2 & true   & \\
%         QBDF2 & true   & \\
%         QNDF  & false  & Erros DimensionMissmatch \\
%         QBDF  & false  & Erros DimensionMismatch \\
%         MEBDF2& false  & Works but slow \\
%         FBDF  & false  & Doesn work on gpu because scalar indexing \\
%     \end{tabular}
%     \caption{Mutlistep Methods}
%     \label{tab:my_label}
% \end{table}

% \begin{table}[]
%     \centering

%     \begin{tabular}{p{5cm}|c|p{5cm}}
%         Name & Funktionale & Fehler \\
%         \hline\hline
%         SSPSDIRK2 & false & Error: MethError: Cannot `convert` an object of type Float32 to an object of type CUDA.CuArray{Float32, 2, CUDA.Mem.DeviceBuffer}║ \\        
%     \end{tabular}
%     \caption{Implicit Strong-Stability Preserving Runge-Kutta Methods for Hyperbolic PDEs}
%     \label{tab:my_label}
% \end{table}

