\chapter*{Kurzfassung}

Diese Arbeit demonstriert wie es möglich ist Implizite numerische differenzialgleichungslöser 
auf der Graphikkarte für Meshgraphnets zu verwenden.
Ein Meshgraphnet besteht dabei aus drei komponenten.
Einer Gitterstruktur die die diskreten messpunkte der simulation darstellen.
Einem Graphen der dazu genutz wird informationen zwischen den einzelnen gitter punkten im Mesh auszutauschen
und einen NeuronalenNetzt das den nächsten zeit schritt vorher sagt.
Um das Neuronale Netz auswerten zukönnen soll ein impliziter differential gleichungs löser verwendet werden.
Das lösen von implizitern verfahren setzt jedoch vorraus das die implementierung des neuronalen
netzen automatisch abgeleitet werden kann.
Innerhalb der implementierung des Meshgraphnets wird die scatter funktion verwendet um informationen
innerhalb der graphdarstellung zwischen den einzelnen konoten zupropagieren.
Diese kann jedoch keine dual zahlen verarbeiten, welche grundlage für die automatischen differenzierung sind.
In dieser Arbeit wird demonstriert wie diese Funktion so modifiziert werden kann das dies doch möglich ist.
Annschließend wird der effekt der impliziten lösen auf die genauigkeit und laufzeit der Meshgraphnets betrachtet. 

\cleardoublepage