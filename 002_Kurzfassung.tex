\chapter*{Kurzfassung}

Diese Arbeit demonstriert, wie es möglich ist implizite Differenzialgleichungslöser 
auf der Graphikkarte für \textit{MeshGraphNets} zu verwenden.
Ein \textit{MeshGraphNet} besteht dabei aus drei Komponenten.
Einer Gitterstruktur, die die diskreten Messpunkte der Simulation darstellen.
Einem Graphen, der dazu genutzt wird, Informationen zwischen den einzelnen gitter punkten im Mesh auszutauschen
und einem neuronalen Netzwerk, das den nächsten zeit schritt vorher sagt.
Um das neuronale Netzwerk auswerten zu können, soll ein impliziter Differenzialgleichungslöser verwendet werden.
Das Lösen von Impliziten verfahren setzt jedoch voraus, das die Implementierung des Neuronalen
netzen automatisch abgeleitet werden kann.
In der Implementierung des MeshGraphNets wird die scatter Funktion verwendet, um Informationen
innerhalb der Graphdarstellung zwischen den einzelnen Knoten zu propagieren.
Diese kann jedoch keine dual zahlen verarbeiten, welche Grundlage für die Automatischen Differenzierung sind.
In dieser Arbeit wird demonstriert, wie diese Funktion so modifiziert werden kann, dass dies doch möglich ist.
Anschließend wird der Effekt der Impliziten lösen auf die Genauigkeit und Laufzeit der MeshGraphNets betrachtet.

\cleardoublepage