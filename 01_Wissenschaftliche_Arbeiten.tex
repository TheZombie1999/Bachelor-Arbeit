\cleardoublepage
\chapter{Verfassen wissenschaftlicher Arbeiten}\label{ch:Wissenschaftliche_Arbeiten}

\section{Formale Aspekte}\label{sec:Formales}
Folgende formale Aspekte sollte die schriftliche Arbeit erfüllen und werden durch die LaTex-Vorlage bereits umgesetzt:
\begin{itemize}
	\item DIN A4
	\item Vorzugsweise einseitig
	\item Satzspiegel
	\begin{itemize}
		\item Breite: \SI{168}{\milli\meter}
		\item Höhe: \SI{237,6}{\milli\meter}
	\end{itemize}
	\item Ränder
	\begin{itemize}
		\item Oben: \SI{19,8}{\milli\meter}
		\item Innen: \SI{28,0}{\milli\meter}
	\end{itemize}
	\item Schriftgröße: 12 (oder 11)
	\item Neue Kapitel auf neuer Seite beginnen lassen
	\item Neue Kapitel im zweiseitigen Druck rechts beginnen lassen
	\item Blocksatz mit angemessener Silbentrennung
	\item Zeilenabstand: 1,5
	\item Seitenzahlen
	\begin{itemize}
		\item Deckblatt enthält keine Seitenzahl
		\item Seitenzahlen zwischen Deckblatt und erstem Kapitel mit römischer Zahlschrift
		\item Beginnend im ersten Kapitel mit arabischer Zahlschrift
	\end{itemize}
	\item Kopfzeile enthält aktuelles Kapitel bzw. Abschnitt
	\item Zitationsstil: IEEE (\url{https://thesius.de/blog/articles/zitieren-ingenieur-ieee-din-iso-690/})
	\item Der Sprache der Arbeit entsprechende Dezimaltrennung
	\item Der Sprache der Arbeit entsprechende Anführungszeichen
	\item Inhaltsverzeichnis
	\item Tabellenüberschriften
	\item Bildunterschriften
	\item Vektorgrafiken
	\item Umflossene Abbildungen vermeiden
	\item Formeln
	\begin{itemize}
		\item Nummerierung
		\item Formelnummerierung enthält Kapitel und fortlaufende Nummer
		\item Symmetrische Ausrichtung (z.B. am Gleichheitszeichen)
	\end{itemize}
\end{itemize}

\section{Inhaltliche Aspekte}\label{sec:Inhaltliches}

\subsection{Sprachstil}
Grundsätzlich ist die wissenschaftliche Arbeit sachlich zu verfassen. Mit Ausnahme einer möglichen Stellungnahme oder Bewertung eines Verfahrens oder einer Methode ist auf einen objektiven Sprachstil zu achten. Die Formulierungen innerhalb der wissenschaftlichen Arbeit sollten nicht zu kompliziert sein. Mehrere Hauptsätze sind besser lesbar als tief verschachtelte Satzkonstruktionen. Weitschweifige Formulierungen sind zu vermeiden. Die Arbeit sollte nur beinhalten, was zum Verständnis beiträgt. Nominalisierungen behindern den Lesefluss und sollten sparsam eingesetzt werden.
\\ \\
\textit{Beispiel: 	\glqq Die Nutzung von Nominalisierungen fördert eine negative Beeinträchtigung des Leseflusses. \grqq{}}
\\ \\
Ebenso sollte auf das Pronomen 	\glqq man\grqq{} sowie die Verwendung der Ich- bzw. Wir-Perspektive verzichtet werden.
\\ \\
\textit{Beispiel: 	\glqq In der folgenden Abbildung kann man den beschriebenen Umstand nachvollziehen. \grqq{}}
\\
\textit{Beispiel: 	\glqq Wir haben den beschriebenen Umstand in folgender Abbildung darstellt. \grqq{}}
\\ \\
Entsprechende Sätze sollten passiv formuliert werden.
\\ \\
\textit{Beispiel: 	\glqq Der beschriebene Umstand ist in folgender Abbbildung nachvollziehbar. \grqq{}}

\subsection{Überarbeitung des Textes}
Fertige Textbausteine oder Abschnitte sollten separat bzgl. des Inhaltes als auch der Rechtschreibung kontrolliert werden. Es bietet sich an, den Text zunächst Abschnittweise auf seine Sinnhaftigkeit zu prüfen und anschließend (möglicherweise mit etwas \glqq Abstand\grqq{}) \glqq Wort für Wort\grqq{} zu lesen, um beispielsweise Buchstabendreher zu vermeiden. Der wissenschaftliche Betreuer dient nicht der grammatikalischen Korrektur des Textes.

Eine abschließende Korrektur der Arbeit sollte folgende Aspekte abdecken:
\begin{itemize}
    \item Wurden alle Abkürzungen beim ersten Auftreten eingeführt?
    \item Sind Vektoren und Matrizen durchgängig gleich formatiert?
    \item Wird im Text auf alle Abbildungen und Tabellen Bezug genommen?
    \item Laufen Bilder, Tabellen, Texte über die Ränder des Textsatzes?
    \item Treten Tabellen und Bilder innerhalb der zugehörigen Kapitel auf bzw. wurden sie auf separate Seiten gesetzt?
    \item Wurden Vektorgrafiken genutzt bzw. sind alle Bilder nach einem Probedruck gut lesbar?
\end{itemize}

\subsection{Nutzung von Quellen}

Zitieren von Quellen ist ein wesentlicher Bestandteil von wissenschaftlichen Arbeiten, da man sich auf Ergebnisse und Erkenntnisse anderer Wissenschaftler bezieht. Es muss in einer Arbeit stets kenntlich gemacht werden, welche Ideen, Vorgehensweisen und Aussagen übernommen wurden. 

Folgende Aspekte sollten bei der Auswahl der Quellen berücksichtigt werden:
\begin{itemize}
    \item Spezifische Fachliteratur verwenden
    \item Keine Populärzeitschriften, Tageszeitungen
    \item Internetquellen sorgfältig prüfen
    \item gedruckte Publikationen bevorzugt verwenden
    \item möglichst den Originaltext zitieren
\end{itemize}


\textbf{Direkte Zitate}

\begin{itemize}
    \item Direkte Zitate werden in Anführungszeichen gesetzt.
    \item Fehler im direkten Zitat müssen übernommen werden. Man kann diese mit [sic!] (lat. sic für \glqq wirklich so\grqq) kennzeichnen.
    \item Textstellen können Hervorgehoben werden (fett, kursiv, unterstrichen oder gesperrt). Das muss jedoch mit dem Hinweise [Hervorhebung des Verfassers] versehen werden. Hervorhebungen des Originaltextes müssen übernommen werden und mit [Hervorhebung im Original] gekennzeichnet werden.
    \item Auslassungen und grammatikalische Anpassungen werden mit [...] gekennzeichnet.
    \item Erläuterungen können in eckigen Klammern hinzugefügt werden.
\end{itemize}

\textbf{Indirekte Zitate}

\begin{itemize}
    \item Aussage wird in eigenen Worten wiedergegeben
    \item Quellenverweis wird mit \glqq vgl. [Quelle]\grqq{} angegeben
\end{itemize}

\textbf{Quellenangaben}

Für bestimmte Aussagen, Bilder, Tabellen oder auch Formeln sind Quellenverweise notwendig. Diese werden entsprechend dem Zitationsstil eingefügt. Eine Ausführliche Erklärung dazu findet sich hier: \url{https://thesius.de/blog/articles/zitieren-ingenieur-ieee-din-iso-690/}.

\subsection{Abbildungen und Tabellen}

Die Formatierung von Abbildungen und Tabellen soll stets einheitlich wie folgt vorgenommen werden:

\textbf{Abbildungen}

\begin{itemize}
    \item Abbildungen auf der Seite zentriert einfügen.
    \item Nummerierung immer mit vorangestellter Kapitelnummer. Die zweite Abbildung in Kapitel vier heißt demnach \glqq Abbildung 4.2.\grqq.
    \item Eine Bildunterschrift soll das dargestellte kurz beschreiben. Sie befindet sich \textbf{unter} der Abbildung und hat die Nummer der Abbildung vorangestellt (z.B. Abbildung 4.2.: [Beschreibungstext])
    \item Eine Abbildung muss immer auch im Text genannt und erklärt werden.
    \item Quellenverweise für Abbildungen werden in der Bildunterschrift eingefügt.
    \item Von Text umflossene Bilder sollten vermieden werden.
    \item Die Bilder sollten möglichst nahe am verweisenden Text eingefügt werden.
    \item Wenn möglich, sollten Vektorgrafiken (pdf; svg; eps; emf) verwendet werden.
    
\end{itemize}

\textbf{Tabellen}

Tabellen bekommen eine Überschrift, keine Unterschrift. Außerdem sollen keine Screenshots verwendet werden sondern die Tabellen im Dokument erstellt werden. Bei der Erstellung der Arbeit in Word können Excel-Tabellen direkt integriert werden. Ansonsten gelten die gleichen Regeln wie für Bilder. 