\chapter{Programmcode}
Programmcode lässt sich in der \verb+lstlisting+ Umgebung direkt schreiben oder über \verb+\lstinputlisting{PFAD/m_file.m}+ aus einer Datei einfügen. 

\begin{lstlisting}
i = 1; 
if ( i == 1){
	i--;
}
\end{lstlisting}


Verwendet man die definierten Styles \texttt{numbers} und \texttt{nonumbers}, lassen sich auch Zeilennummerierungen hinzufügen. Dazu setzt man \verb+\lstset{style = numbers}+.
\lstset{style = numbers}
\begin{lstlisting}
i = 1; 
if (i == 1){
	i--;
}
\end{lstlisting}

Code-Schnipsel, wie \lstinline!if(i == 1)! lassen sich auch direkt im Fließtext einbauen. Dazu wird \verb+\lstinline!CODE!+ verwendet.
Für Matlab-Code kann hier auch \verb+\mcode{CODE}+ genutzt werden. 

Außerdem kann mit \verb+\lstinputlisting{PFAD/DATEI}+ Code auch direkt aus Matlab-Dateien importiert werden. 
