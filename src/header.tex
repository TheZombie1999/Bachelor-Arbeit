%   header.tex 
%   Version 1.0     |   Peter Krönes    |   08.05.2018
%%%%%%%%%%%%%%%%%%%%%%%%%%%%%%%%%%%%%%%%%%%%%%%%%%%%%%%%%%%%%%%%%%%%%%%%
\documentclass
[   oneside,         % oneside/twoside : Einseitiger oder zweiseitiger Druck?
    12pt,            % Bezug: 12-Punkt Schriftgröße
    DIV=15,          % Randaufteilung, siehe Dokumentation "KOMA"-Script
    BCOR=17mm,       % Bindekorrektur: Innen 17mm Platz lassen. Copyshop-getestet.
    headsepline,     % Unter Kopfzeile Trennlinie (aus: headnosepline)
    openright,       % Neue Kapitel im zweiseitigen Druck rechts beginnen lassen
    a4paper,         % Seitenformat A4
    listof=totoc,      % Div. Verzeichnisse ins Inhaltsverzeichnis aufnehmen
    bibliography=totoc,        % Literaturverzeichnis ins Inhaltsverzeichnis aufnehmen
]   {scrbook}        %scrbook für Abschlussarbeiten 
%%%%%%%%%%%%%%%%%%%%%%%%%%%%%%%%%%%%%%%%%%%%%%%%%%%%%%%%%%%%%%%%%%%%%%%%%%%%%
\usepackage[
    backend=biber,
    natbib=true,
    url=false, 
    doi=true,
    eprint=false,
    backref=true
]{biblatex}
\usepackage{hyperref}%[pdftex,bookmarks,colorlinks,citecolor=black,breakl inks]
\appto\UrlBreaks{\do\a\do\b\do\c\do\d\do\e\do\f\do\g\do\h\do\i\do\j
	\do\k\do\l\do\m\do\n\do\o\do\p\do\q\do\r\do\s\do\t\do\u\do\v\do\w
	\do\x\do\y\do\z}						% Umbruch von langen URLs

\DefineBibliographyStrings{ngerman}{%
  backrefpage = {S.},                       % originally "cited on page"
  backrefpages = {S.},                      % originally "cited on pages"
  andothers = {{et\,al\adddot}},
}
%%%%%%%%%%%%%%%%%%%%%         PAKETE        %%%%%%%%%%%%%%%%%%%%%%%%%%%%
\usepackage[utf8]{inputenc} 
\usepackage{csquotes}                       % Anführungszeichen
\usepackage{setspace}                       % Zeilenabstand
\usepackage[ngerman]{babel}                 % Spracheinstellung
\usepackage{float}                          % Platzierung von Objekten
\floatstyle{plaintop}						% Caption über Tabellen
\restylefloat{table}						% ^^^^^^^^^^^^^^^^^^^^
\usepackage{pdfpages}                       % PDF Seitenweise einfügen
\usepackage[font=normalsize]{caption}       % Größe der Bild-/Tabellenunterschrift 
\usepackage{acronym}                        % Abkürzungsverzeichnis
\newcommand{\acrosecondcolumn}[1]{
  \acroextra{\makebox[70mm][l]{#1}}
}
\usepackage{graphicx}                       % Für Bilder
\usepackage{subcaption}                     % mehrere Bilder nebeneinander
\usepackage{floatflt}                       % Umlossene Bilder
%\usepackage{overpic}						% Überlagerung von Bildern
\usepackage{svg}                            % Vektorgrafiken im svg-format
%\usepackage{tabu}						    % Abstände in Tabellen (\tabulinesep)
\usepackage{tabularx}                       % Für Tabellen
\usepackage{booktabs}                       % Noch ein Tabellenpaket
\usepackage{color}                          % farbiger Text
\usepackage{lineno}                         % Zeilennummern 
\usepackage{microtype}                      % Macht alles schöner
\usepackage{listings}                       % Für Code
\usepackage{src/mcode/mcode}                % Matlab-Code Paket
\usepackage{src/abkuerzung}                 % Abkürzungen
\usepackage{blindtext}                      % Beispieltext zum auspobieren
\usepackage{todonotes}                      % Fügt Todo Notes ein
\usepackage{eurosym}                        % Eurozeichen mit \euro
\usepackage{afterpage}						% A3 pages in A4 Document
\usepackage[decimalsymbol=comma]{siunitx}   % SI-Einheiten
\usepackage{array}                          % Array


%% Tikz-Pakete
\usepackage{tikz}
\usepackage[siunitx,european,straightvoltages]{circuitikz}          % Schaltkreise
\usetikzlibrary{decorations.pathreplacing}                          % Schaltkreiszusatz
%\usetikzlibrary{arrows.meta}
\usetikzlibrary{arrows}
\usetikzlibrary{patterns}
\usetikzlibrary{ipe}
\usetikzlibrary{positioning}
%\usetikzlibrary{shapes.geometric}
\usetikzlibrary{shapes}
\usetikzlibrary{backgrounds}

%%Mathematik-Pakete für Formeln
\usepackage{amsmath}
\usepackage{amssymb}
\usepackage{amstext}
\usepackage{amsfonts}
\usepackage{mathrsfs}
\usepackage{mathtools}
%\usepackage{esvect}
%\numberwithin{equation}{section} %Nummerierungsebene für Gleichungen
\usepackage{gensymb}
\usepackage{multirow}
\usepackage{bigdelim}
\usepackage{epsfig}
\usepackage{etoolbox} 
\makeatletter 
\patchcmd\Gread@eps{\@inputcheck#1 }{\@inputcheck"#1"\relax}{}{}
\makeatother
%\renewcommand*{\familydefault}{\sfdefault}
%\newcommand*{\head}{\bfseries}
%
\newcolumntype{_}{>{\global\let\currentrowstyle\relax}}
\newcolumntype{^}{>{\currentrowstyle}}
\newcommand{\rowstyle}[1]{\gdef\currentrowstyle{#1}%
	#1\ignorespaces
}

%%%%%%%%%%%%%%%%%%%%%%%%%%%%%%%%%%%%%%%%%%%%%%%%%%%%%%%%%%%%%%%%%%%%%%%% 
% Setzt die Kapitel im Anhang nicht ins Inhaltsverzeichnis, setzt man 
% Abbildungsverzeichnis und Tabellenverzeichnis hinter den Anhang erscheinen 
% diese Trotzdem im Inhaltsverzeichnis.
%\usepackage[toc,page]{appendix}
%%%%%%%%%%%%%%%%%%%%%%%%%%%%%%%%%%%%%%%%%%%%%%%%%%%%%%%%%%%%%%%%%%%%%%%%
\setlength{\parindent}{0pt}                 % Absätze nicht eingerückt
%% Änderungen Pauline %%%%%%%%%%%%%%%%%%%%%%
\usepackage{cleveref}						% ans Ende verschoben, da Kompilierungsfehler

% Styles um Code mit oder ohne Zeilennummerierung auszugeben
\lstdefinestyle{numbers}{numbers=left, stepnumber=1, numberstyle=\tiny, numbersep=10pt,xleftmargin=1.5em}
\lstdefinestyle{nonumbers}{numbers=none}

%% Glossaries für Symbolverzeichnis
\usepackage[acronym,automake]{glossaries}             
\usepackage{glossary-longbooktabs}

\newglossary[klg]{konstante}{koi}{kog}{Konstanten} 
\newglossary[slg]{symbol}{syi}{syg}{Symbole} 
\glsaddkey{unit}{\glsentrytext{\glslabel}}{\glsentryunit}{\GLsentryunit}{\glsunit}{\Glsunit}{\GLSunit}

\newglossarystyle{3colger}{%
    \setglossarystyle{longragged3col}% base this style on the list style
    \renewenvironment{theglossary}{% Change the table type --> 3 columns
     % compute the description width
        \settowidth{\dimen0}{Zeichen}%  % so viel Platz, wie "Zeichen" braucht
        \settowidth{\dimen2}{laaaaangeEinheit}%
        \setlength{\glsdescwidth}{\linewidth-\dimen0-\dimen2-6\tabcolsep}
        \begin{longtable}{p{\dimen0} p{\glsdescwidth} p{\dimen2}}}%
        {\end{longtable}}%
    %
% \renewcommand*{\glossaryheader}{%  Change the table header
%  \bfseries Zeichen & \bfseries Beschreibung & \bfseries Einheit \\
%           \hline
%   \vspace{0.05cm}
%   \endhead}
\renewcommand*{\glossentry}[2]{%  Change the displayed items
    \glstarget{##1}{\glossentryname{##1}} %
    &  \glossentrydesc{##1}
    & \glsunit{##1}  \tabularnewline
}
\renewcommand*{\glsclearpage}{}  % damit alles auf einer Seite bleibt
}
\makeglossaries                                   % activate glossaries-package
\newcommand{\symbolverzeichnis}{
\glsaddall
\setglossarysection{section}
\printglossary[type = symbol, style=3colger]
\printglossary[type = konstante, style=3colger]}